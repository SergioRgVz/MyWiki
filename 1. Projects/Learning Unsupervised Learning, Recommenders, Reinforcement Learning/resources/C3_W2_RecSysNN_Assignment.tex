\documentclass[11pt]{article}

    \usepackage[breakable]{tcolorbox}
    \usepackage{parskip} % Stop auto-indenting (to mimic markdown behaviour)
    
    \usepackage{iftex}
    \ifPDFTeX
    	\usepackage[T1]{fontenc}
    	\usepackage{mathpazo}
    \else
    	\usepackage{fontspec}
    \fi

    % Basic figure setup, for now with no caption control since it's done
    % automatically by Pandoc (which extracts ![](path) syntax from Markdown).
    \usepackage{graphicx}
    % Maintain compatibility with old templates. Remove in nbconvert 6.0
    \let\Oldincludegraphics\includegraphics
    % Ensure that by default, figures have no caption (until we provide a
    % proper Figure object with a Caption API and a way to capture that
    % in the conversion process - todo).
    \usepackage{caption}
    \DeclareCaptionFormat{nocaption}{}
    \captionsetup{format=nocaption,aboveskip=0pt,belowskip=0pt}

    \usepackage[Export]{adjustbox} % Used to constrain images to a maximum size
    \adjustboxset{max size={0.9\linewidth}{0.9\paperheight}}
    \usepackage{float}
    \floatplacement{figure}{H} % forces figures to be placed at the correct location
    \usepackage{xcolor} % Allow colors to be defined
    \usepackage{enumerate} % Needed for markdown enumerations to work
    \usepackage{geometry} % Used to adjust the document margins
    \usepackage{amsmath} % Equations
    \usepackage{amssymb} % Equations
    \usepackage{textcomp} % defines textquotesingle
    % Hack from http://tex.stackexchange.com/a/47451/13684:
    \AtBeginDocument{%
        \def\PYZsq{\textquotesingle}% Upright quotes in Pygmentized code
    }
    \usepackage{upquote} % Upright quotes for verbatim code
    \usepackage{eurosym} % defines \euro
    \usepackage[mathletters]{ucs} % Extended unicode (utf-8) support
    \usepackage{fancyvrb} % verbatim replacement that allows latex
    \usepackage{grffile} % extends the file name processing of package graphics 
                         % to support a larger range
    \makeatletter % fix for grffile with XeLaTeX
    \def\Gread@@xetex#1{%
      \IfFileExists{"\Gin@base".bb}%
      {\Gread@eps{\Gin@base.bb}}%
      {\Gread@@xetex@aux#1}%
    }
    \makeatother

    % The hyperref package gives us a pdf with properly built
    % internal navigation ('pdf bookmarks' for the table of contents,
    % internal cross-reference links, web links for URLs, etc.)
    \usepackage{hyperref}
    % The default LaTeX title has an obnoxious amount of whitespace. By default,
    % titling removes some of it. It also provides customization options.
    \usepackage{titling}
    \usepackage{longtable} % longtable support required by pandoc >1.10
    \usepackage{booktabs}  % table support for pandoc > 1.12.2
    \usepackage[inline]{enumitem} % IRkernel/repr support (it uses the enumerate* environment)
    \usepackage[normalem]{ulem} % ulem is needed to support strikethroughs (\sout)
                                % normalem makes italics be italics, not underlines
    \usepackage{mathrsfs}
    

    
    % Colors for the hyperref package
    \definecolor{urlcolor}{rgb}{0,.145,.698}
    \definecolor{linkcolor}{rgb}{.71,0.21,0.01}
    \definecolor{citecolor}{rgb}{.12,.54,.11}

    % ANSI colors
    \definecolor{ansi-black}{HTML}{3E424D}
    \definecolor{ansi-black-intense}{HTML}{282C36}
    \definecolor{ansi-red}{HTML}{E75C58}
    \definecolor{ansi-red-intense}{HTML}{B22B31}
    \definecolor{ansi-green}{HTML}{00A250}
    \definecolor{ansi-green-intense}{HTML}{007427}
    \definecolor{ansi-yellow}{HTML}{DDB62B}
    \definecolor{ansi-yellow-intense}{HTML}{B27D12}
    \definecolor{ansi-blue}{HTML}{208FFB}
    \definecolor{ansi-blue-intense}{HTML}{0065CA}
    \definecolor{ansi-magenta}{HTML}{D160C4}
    \definecolor{ansi-magenta-intense}{HTML}{A03196}
    \definecolor{ansi-cyan}{HTML}{60C6C8}
    \definecolor{ansi-cyan-intense}{HTML}{258F8F}
    \definecolor{ansi-white}{HTML}{C5C1B4}
    \definecolor{ansi-white-intense}{HTML}{A1A6B2}
    \definecolor{ansi-default-inverse-fg}{HTML}{FFFFFF}
    \definecolor{ansi-default-inverse-bg}{HTML}{000000}

    % commands and environments needed by pandoc snippets
    % extracted from the output of `pandoc -s`
    \providecommand{\tightlist}{%
      \setlength{\itemsep}{0pt}\setlength{\parskip}{0pt}}
    \DefineVerbatimEnvironment{Highlighting}{Verbatim}{commandchars=\\\{\}}
    % Add ',fontsize=\small' for more characters per line
    \newenvironment{Shaded}{}{}
    \newcommand{\KeywordTok}[1]{\textcolor[rgb]{0.00,0.44,0.13}{\textbf{{#1}}}}
    \newcommand{\DataTypeTok}[1]{\textcolor[rgb]{0.56,0.13,0.00}{{#1}}}
    \newcommand{\DecValTok}[1]{\textcolor[rgb]{0.25,0.63,0.44}{{#1}}}
    \newcommand{\BaseNTok}[1]{\textcolor[rgb]{0.25,0.63,0.44}{{#1}}}
    \newcommand{\FloatTok}[1]{\textcolor[rgb]{0.25,0.63,0.44}{{#1}}}
    \newcommand{\CharTok}[1]{\textcolor[rgb]{0.25,0.44,0.63}{{#1}}}
    \newcommand{\StringTok}[1]{\textcolor[rgb]{0.25,0.44,0.63}{{#1}}}
    \newcommand{\CommentTok}[1]{\textcolor[rgb]{0.38,0.63,0.69}{\textit{{#1}}}}
    \newcommand{\OtherTok}[1]{\textcolor[rgb]{0.00,0.44,0.13}{{#1}}}
    \newcommand{\AlertTok}[1]{\textcolor[rgb]{1.00,0.00,0.00}{\textbf{{#1}}}}
    \newcommand{\FunctionTok}[1]{\textcolor[rgb]{0.02,0.16,0.49}{{#1}}}
    \newcommand{\RegionMarkerTok}[1]{{#1}}
    \newcommand{\ErrorTok}[1]{\textcolor[rgb]{1.00,0.00,0.00}{\textbf{{#1}}}}
    \newcommand{\NormalTok}[1]{{#1}}
    
    % Additional commands for more recent versions of Pandoc
    \newcommand{\ConstantTok}[1]{\textcolor[rgb]{0.53,0.00,0.00}{{#1}}}
    \newcommand{\SpecialCharTok}[1]{\textcolor[rgb]{0.25,0.44,0.63}{{#1}}}
    \newcommand{\VerbatimStringTok}[1]{\textcolor[rgb]{0.25,0.44,0.63}{{#1}}}
    \newcommand{\SpecialStringTok}[1]{\textcolor[rgb]{0.73,0.40,0.53}{{#1}}}
    \newcommand{\ImportTok}[1]{{#1}}
    \newcommand{\DocumentationTok}[1]{\textcolor[rgb]{0.73,0.13,0.13}{\textit{{#1}}}}
    \newcommand{\AnnotationTok}[1]{\textcolor[rgb]{0.38,0.63,0.69}{\textbf{\textit{{#1}}}}}
    \newcommand{\CommentVarTok}[1]{\textcolor[rgb]{0.38,0.63,0.69}{\textbf{\textit{{#1}}}}}
    \newcommand{\VariableTok}[1]{\textcolor[rgb]{0.10,0.09,0.49}{{#1}}}
    \newcommand{\ControlFlowTok}[1]{\textcolor[rgb]{0.00,0.44,0.13}{\textbf{{#1}}}}
    \newcommand{\OperatorTok}[1]{\textcolor[rgb]{0.40,0.40,0.40}{{#1}}}
    \newcommand{\BuiltInTok}[1]{{#1}}
    \newcommand{\ExtensionTok}[1]{{#1}}
    \newcommand{\PreprocessorTok}[1]{\textcolor[rgb]{0.74,0.48,0.00}{{#1}}}
    \newcommand{\AttributeTok}[1]{\textcolor[rgb]{0.49,0.56,0.16}{{#1}}}
    \newcommand{\InformationTok}[1]{\textcolor[rgb]{0.38,0.63,0.69}{\textbf{\textit{{#1}}}}}
    \newcommand{\WarningTok}[1]{\textcolor[rgb]{0.38,0.63,0.69}{\textbf{\textit{{#1}}}}}
    
    
    % Define a nice break command that doesn't care if a line doesn't already
    % exist.
    \def\br{\hspace*{\fill} \\* }
    % Math Jax compatibility definitions
    \def\gt{>}
    \def\lt{<}
    \let\Oldtex\TeX
    \let\Oldlatex\LaTeX
    \renewcommand{\TeX}{\textrm{\Oldtex}}
    \renewcommand{\LaTeX}{\textrm{\Oldlatex}}
    % Document parameters
    % Document title
    \title{C3\_W2\_RecSysNN\_Assignment}
    
    
    
    
    
% Pygments definitions
\makeatletter
\def\PY@reset{\let\PY@it=\relax \let\PY@bf=\relax%
    \let\PY@ul=\relax \let\PY@tc=\relax%
    \let\PY@bc=\relax \let\PY@ff=\relax}
\def\PY@tok#1{\csname PY@tok@#1\endcsname}
\def\PY@toks#1+{\ifx\relax#1\empty\else%
    \PY@tok{#1}\expandafter\PY@toks\fi}
\def\PY@do#1{\PY@bc{\PY@tc{\PY@ul{%
    \PY@it{\PY@bf{\PY@ff{#1}}}}}}}
\def\PY#1#2{\PY@reset\PY@toks#1+\relax+\PY@do{#2}}

\expandafter\def\csname PY@tok@w\endcsname{\def\PY@tc##1{\textcolor[rgb]{0.73,0.73,0.73}{##1}}}
\expandafter\def\csname PY@tok@c\endcsname{\let\PY@it=\textit\def\PY@tc##1{\textcolor[rgb]{0.25,0.50,0.50}{##1}}}
\expandafter\def\csname PY@tok@cp\endcsname{\def\PY@tc##1{\textcolor[rgb]{0.74,0.48,0.00}{##1}}}
\expandafter\def\csname PY@tok@k\endcsname{\let\PY@bf=\textbf\def\PY@tc##1{\textcolor[rgb]{0.00,0.50,0.00}{##1}}}
\expandafter\def\csname PY@tok@kp\endcsname{\def\PY@tc##1{\textcolor[rgb]{0.00,0.50,0.00}{##1}}}
\expandafter\def\csname PY@tok@kt\endcsname{\def\PY@tc##1{\textcolor[rgb]{0.69,0.00,0.25}{##1}}}
\expandafter\def\csname PY@tok@o\endcsname{\def\PY@tc##1{\textcolor[rgb]{0.40,0.40,0.40}{##1}}}
\expandafter\def\csname PY@tok@ow\endcsname{\let\PY@bf=\textbf\def\PY@tc##1{\textcolor[rgb]{0.67,0.13,1.00}{##1}}}
\expandafter\def\csname PY@tok@nb\endcsname{\def\PY@tc##1{\textcolor[rgb]{0.00,0.50,0.00}{##1}}}
\expandafter\def\csname PY@tok@nf\endcsname{\def\PY@tc##1{\textcolor[rgb]{0.00,0.00,1.00}{##1}}}
\expandafter\def\csname PY@tok@nc\endcsname{\let\PY@bf=\textbf\def\PY@tc##1{\textcolor[rgb]{0.00,0.00,1.00}{##1}}}
\expandafter\def\csname PY@tok@nn\endcsname{\let\PY@bf=\textbf\def\PY@tc##1{\textcolor[rgb]{0.00,0.00,1.00}{##1}}}
\expandafter\def\csname PY@tok@ne\endcsname{\let\PY@bf=\textbf\def\PY@tc##1{\textcolor[rgb]{0.82,0.25,0.23}{##1}}}
\expandafter\def\csname PY@tok@nv\endcsname{\def\PY@tc##1{\textcolor[rgb]{0.10,0.09,0.49}{##1}}}
\expandafter\def\csname PY@tok@no\endcsname{\def\PY@tc##1{\textcolor[rgb]{0.53,0.00,0.00}{##1}}}
\expandafter\def\csname PY@tok@nl\endcsname{\def\PY@tc##1{\textcolor[rgb]{0.63,0.63,0.00}{##1}}}
\expandafter\def\csname PY@tok@ni\endcsname{\let\PY@bf=\textbf\def\PY@tc##1{\textcolor[rgb]{0.60,0.60,0.60}{##1}}}
\expandafter\def\csname PY@tok@na\endcsname{\def\PY@tc##1{\textcolor[rgb]{0.49,0.56,0.16}{##1}}}
\expandafter\def\csname PY@tok@nt\endcsname{\let\PY@bf=\textbf\def\PY@tc##1{\textcolor[rgb]{0.00,0.50,0.00}{##1}}}
\expandafter\def\csname PY@tok@nd\endcsname{\def\PY@tc##1{\textcolor[rgb]{0.67,0.13,1.00}{##1}}}
\expandafter\def\csname PY@tok@s\endcsname{\def\PY@tc##1{\textcolor[rgb]{0.73,0.13,0.13}{##1}}}
\expandafter\def\csname PY@tok@sd\endcsname{\let\PY@it=\textit\def\PY@tc##1{\textcolor[rgb]{0.73,0.13,0.13}{##1}}}
\expandafter\def\csname PY@tok@si\endcsname{\let\PY@bf=\textbf\def\PY@tc##1{\textcolor[rgb]{0.73,0.40,0.53}{##1}}}
\expandafter\def\csname PY@tok@se\endcsname{\let\PY@bf=\textbf\def\PY@tc##1{\textcolor[rgb]{0.73,0.40,0.13}{##1}}}
\expandafter\def\csname PY@tok@sr\endcsname{\def\PY@tc##1{\textcolor[rgb]{0.73,0.40,0.53}{##1}}}
\expandafter\def\csname PY@tok@ss\endcsname{\def\PY@tc##1{\textcolor[rgb]{0.10,0.09,0.49}{##1}}}
\expandafter\def\csname PY@tok@sx\endcsname{\def\PY@tc##1{\textcolor[rgb]{0.00,0.50,0.00}{##1}}}
\expandafter\def\csname PY@tok@m\endcsname{\def\PY@tc##1{\textcolor[rgb]{0.40,0.40,0.40}{##1}}}
\expandafter\def\csname PY@tok@gh\endcsname{\let\PY@bf=\textbf\def\PY@tc##1{\textcolor[rgb]{0.00,0.00,0.50}{##1}}}
\expandafter\def\csname PY@tok@gu\endcsname{\let\PY@bf=\textbf\def\PY@tc##1{\textcolor[rgb]{0.50,0.00,0.50}{##1}}}
\expandafter\def\csname PY@tok@gd\endcsname{\def\PY@tc##1{\textcolor[rgb]{0.63,0.00,0.00}{##1}}}
\expandafter\def\csname PY@tok@gi\endcsname{\def\PY@tc##1{\textcolor[rgb]{0.00,0.63,0.00}{##1}}}
\expandafter\def\csname PY@tok@gr\endcsname{\def\PY@tc##1{\textcolor[rgb]{1.00,0.00,0.00}{##1}}}
\expandafter\def\csname PY@tok@ge\endcsname{\let\PY@it=\textit}
\expandafter\def\csname PY@tok@gs\endcsname{\let\PY@bf=\textbf}
\expandafter\def\csname PY@tok@gp\endcsname{\let\PY@bf=\textbf\def\PY@tc##1{\textcolor[rgb]{0.00,0.00,0.50}{##1}}}
\expandafter\def\csname PY@tok@go\endcsname{\def\PY@tc##1{\textcolor[rgb]{0.53,0.53,0.53}{##1}}}
\expandafter\def\csname PY@tok@gt\endcsname{\def\PY@tc##1{\textcolor[rgb]{0.00,0.27,0.87}{##1}}}
\expandafter\def\csname PY@tok@err\endcsname{\def\PY@bc##1{\setlength{\fboxsep}{0pt}\fcolorbox[rgb]{1.00,0.00,0.00}{1,1,1}{\strut ##1}}}
\expandafter\def\csname PY@tok@kc\endcsname{\let\PY@bf=\textbf\def\PY@tc##1{\textcolor[rgb]{0.00,0.50,0.00}{##1}}}
\expandafter\def\csname PY@tok@kd\endcsname{\let\PY@bf=\textbf\def\PY@tc##1{\textcolor[rgb]{0.00,0.50,0.00}{##1}}}
\expandafter\def\csname PY@tok@kn\endcsname{\let\PY@bf=\textbf\def\PY@tc##1{\textcolor[rgb]{0.00,0.50,0.00}{##1}}}
\expandafter\def\csname PY@tok@kr\endcsname{\let\PY@bf=\textbf\def\PY@tc##1{\textcolor[rgb]{0.00,0.50,0.00}{##1}}}
\expandafter\def\csname PY@tok@bp\endcsname{\def\PY@tc##1{\textcolor[rgb]{0.00,0.50,0.00}{##1}}}
\expandafter\def\csname PY@tok@fm\endcsname{\def\PY@tc##1{\textcolor[rgb]{0.00,0.00,1.00}{##1}}}
\expandafter\def\csname PY@tok@vc\endcsname{\def\PY@tc##1{\textcolor[rgb]{0.10,0.09,0.49}{##1}}}
\expandafter\def\csname PY@tok@vg\endcsname{\def\PY@tc##1{\textcolor[rgb]{0.10,0.09,0.49}{##1}}}
\expandafter\def\csname PY@tok@vi\endcsname{\def\PY@tc##1{\textcolor[rgb]{0.10,0.09,0.49}{##1}}}
\expandafter\def\csname PY@tok@vm\endcsname{\def\PY@tc##1{\textcolor[rgb]{0.10,0.09,0.49}{##1}}}
\expandafter\def\csname PY@tok@sa\endcsname{\def\PY@tc##1{\textcolor[rgb]{0.73,0.13,0.13}{##1}}}
\expandafter\def\csname PY@tok@sb\endcsname{\def\PY@tc##1{\textcolor[rgb]{0.73,0.13,0.13}{##1}}}
\expandafter\def\csname PY@tok@sc\endcsname{\def\PY@tc##1{\textcolor[rgb]{0.73,0.13,0.13}{##1}}}
\expandafter\def\csname PY@tok@dl\endcsname{\def\PY@tc##1{\textcolor[rgb]{0.73,0.13,0.13}{##1}}}
\expandafter\def\csname PY@tok@s2\endcsname{\def\PY@tc##1{\textcolor[rgb]{0.73,0.13,0.13}{##1}}}
\expandafter\def\csname PY@tok@sh\endcsname{\def\PY@tc##1{\textcolor[rgb]{0.73,0.13,0.13}{##1}}}
\expandafter\def\csname PY@tok@s1\endcsname{\def\PY@tc##1{\textcolor[rgb]{0.73,0.13,0.13}{##1}}}
\expandafter\def\csname PY@tok@mb\endcsname{\def\PY@tc##1{\textcolor[rgb]{0.40,0.40,0.40}{##1}}}
\expandafter\def\csname PY@tok@mf\endcsname{\def\PY@tc##1{\textcolor[rgb]{0.40,0.40,0.40}{##1}}}
\expandafter\def\csname PY@tok@mh\endcsname{\def\PY@tc##1{\textcolor[rgb]{0.40,0.40,0.40}{##1}}}
\expandafter\def\csname PY@tok@mi\endcsname{\def\PY@tc##1{\textcolor[rgb]{0.40,0.40,0.40}{##1}}}
\expandafter\def\csname PY@tok@il\endcsname{\def\PY@tc##1{\textcolor[rgb]{0.40,0.40,0.40}{##1}}}
\expandafter\def\csname PY@tok@mo\endcsname{\def\PY@tc##1{\textcolor[rgb]{0.40,0.40,0.40}{##1}}}
\expandafter\def\csname PY@tok@ch\endcsname{\let\PY@it=\textit\def\PY@tc##1{\textcolor[rgb]{0.25,0.50,0.50}{##1}}}
\expandafter\def\csname PY@tok@cm\endcsname{\let\PY@it=\textit\def\PY@tc##1{\textcolor[rgb]{0.25,0.50,0.50}{##1}}}
\expandafter\def\csname PY@tok@cpf\endcsname{\let\PY@it=\textit\def\PY@tc##1{\textcolor[rgb]{0.25,0.50,0.50}{##1}}}
\expandafter\def\csname PY@tok@c1\endcsname{\let\PY@it=\textit\def\PY@tc##1{\textcolor[rgb]{0.25,0.50,0.50}{##1}}}
\expandafter\def\csname PY@tok@cs\endcsname{\let\PY@it=\textit\def\PY@tc##1{\textcolor[rgb]{0.25,0.50,0.50}{##1}}}

\def\PYZbs{\char`\\}
\def\PYZus{\char`\_}
\def\PYZob{\char`\{}
\def\PYZcb{\char`\}}
\def\PYZca{\char`\^}
\def\PYZam{\char`\&}
\def\PYZlt{\char`\<}
\def\PYZgt{\char`\>}
\def\PYZsh{\char`\#}
\def\PYZpc{\char`\%}
\def\PYZdl{\char`\$}
\def\PYZhy{\char`\-}
\def\PYZsq{\char`\'}
\def\PYZdq{\char`\"}
\def\PYZti{\char`\~}
% for compatibility with earlier versions
\def\PYZat{@}
\def\PYZlb{[}
\def\PYZrb{]}
\makeatother


    % For linebreaks inside Verbatim environment from package fancyvrb. 
    \makeatletter
        \newbox\Wrappedcontinuationbox 
        \newbox\Wrappedvisiblespacebox 
        \newcommand*\Wrappedvisiblespace {\textcolor{red}{\textvisiblespace}} 
        \newcommand*\Wrappedcontinuationsymbol {\textcolor{red}{\llap{\tiny$\m@th\hookrightarrow$}}} 
        \newcommand*\Wrappedcontinuationindent {3ex } 
        \newcommand*\Wrappedafterbreak {\kern\Wrappedcontinuationindent\copy\Wrappedcontinuationbox} 
        % Take advantage of the already applied Pygments mark-up to insert 
        % potential linebreaks for TeX processing. 
        %        {, <, #, %, $, ' and ": go to next line. 
        %        _, }, ^, &, >, - and ~: stay at end of broken line. 
        % Use of \textquotesingle for straight quote. 
        \newcommand*\Wrappedbreaksatspecials {% 
            \def\PYGZus{\discretionary{\char`\_}{\Wrappedafterbreak}{\char`\_}}% 
            \def\PYGZob{\discretionary{}{\Wrappedafterbreak\char`\{}{\char`\{}}% 
            \def\PYGZcb{\discretionary{\char`\}}{\Wrappedafterbreak}{\char`\}}}% 
            \def\PYGZca{\discretionary{\char`\^}{\Wrappedafterbreak}{\char`\^}}% 
            \def\PYGZam{\discretionary{\char`\&}{\Wrappedafterbreak}{\char`\&}}% 
            \def\PYGZlt{\discretionary{}{\Wrappedafterbreak\char`\<}{\char`\<}}% 
            \def\PYGZgt{\discretionary{\char`\>}{\Wrappedafterbreak}{\char`\>}}% 
            \def\PYGZsh{\discretionary{}{\Wrappedafterbreak\char`\#}{\char`\#}}% 
            \def\PYGZpc{\discretionary{}{\Wrappedafterbreak\char`\%}{\char`\%}}% 
            \def\PYGZdl{\discretionary{}{\Wrappedafterbreak\char`\$}{\char`\$}}% 
            \def\PYGZhy{\discretionary{\char`\-}{\Wrappedafterbreak}{\char`\-}}% 
            \def\PYGZsq{\discretionary{}{\Wrappedafterbreak\textquotesingle}{\textquotesingle}}% 
            \def\PYGZdq{\discretionary{}{\Wrappedafterbreak\char`\"}{\char`\"}}% 
            \def\PYGZti{\discretionary{\char`\~}{\Wrappedafterbreak}{\char`\~}}% 
        } 
        % Some characters . , ; ? ! / are not pygmentized. 
        % This macro makes them "active" and they will insert potential linebreaks 
        \newcommand*\Wrappedbreaksatpunct {% 
            \lccode`\~`\.\lowercase{\def~}{\discretionary{\hbox{\char`\.}}{\Wrappedafterbreak}{\hbox{\char`\.}}}% 
            \lccode`\~`\,\lowercase{\def~}{\discretionary{\hbox{\char`\,}}{\Wrappedafterbreak}{\hbox{\char`\,}}}% 
            \lccode`\~`\;\lowercase{\def~}{\discretionary{\hbox{\char`\;}}{\Wrappedafterbreak}{\hbox{\char`\;}}}% 
            \lccode`\~`\:\lowercase{\def~}{\discretionary{\hbox{\char`\:}}{\Wrappedafterbreak}{\hbox{\char`\:}}}% 
            \lccode`\~`\?\lowercase{\def~}{\discretionary{\hbox{\char`\?}}{\Wrappedafterbreak}{\hbox{\char`\?}}}% 
            \lccode`\~`\!\lowercase{\def~}{\discretionary{\hbox{\char`\!}}{\Wrappedafterbreak}{\hbox{\char`\!}}}% 
            \lccode`\~`\/\lowercase{\def~}{\discretionary{\hbox{\char`\/}}{\Wrappedafterbreak}{\hbox{\char`\/}}}% 
            \catcode`\.\active
            \catcode`\,\active 
            \catcode`\;\active
            \catcode`\:\active
            \catcode`\?\active
            \catcode`\!\active
            \catcode`\/\active 
            \lccode`\~`\~ 	
        }
    \makeatother

    \let\OriginalVerbatim=\Verbatim
    \makeatletter
    \renewcommand{\Verbatim}[1][1]{%
        %\parskip\z@skip
        \sbox\Wrappedcontinuationbox {\Wrappedcontinuationsymbol}%
        \sbox\Wrappedvisiblespacebox {\FV@SetupFont\Wrappedvisiblespace}%
        \def\FancyVerbFormatLine ##1{\hsize\linewidth
            \vtop{\raggedright\hyphenpenalty\z@\exhyphenpenalty\z@
                \doublehyphendemerits\z@\finalhyphendemerits\z@
                \strut ##1\strut}%
        }%
        % If the linebreak is at a space, the latter will be displayed as visible
        % space at end of first line, and a continuation symbol starts next line.
        % Stretch/shrink are however usually zero for typewriter font.
        \def\FV@Space {%
            \nobreak\hskip\z@ plus\fontdimen3\font minus\fontdimen4\font
            \discretionary{\copy\Wrappedvisiblespacebox}{\Wrappedafterbreak}
            {\kern\fontdimen2\font}%
        }%
        
        % Allow breaks at special characters using \PYG... macros.
        \Wrappedbreaksatspecials
        % Breaks at punctuation characters . , ; ? ! and / need catcode=\active 	
        \OriginalVerbatim[#1,codes*=\Wrappedbreaksatpunct]%
    }
    \makeatother

    % Exact colors from NB
    \definecolor{incolor}{HTML}{303F9F}
    \definecolor{outcolor}{HTML}{D84315}
    \definecolor{cellborder}{HTML}{CFCFCF}
    \definecolor{cellbackground}{HTML}{F7F7F7}
    
    % prompt
    \makeatletter
    \newcommand{\boxspacing}{\kern\kvtcb@left@rule\kern\kvtcb@boxsep}
    \makeatother
    \newcommand{\prompt}[4]{
        \ttfamily\llap{{\color{#2}[#3]:\hspace{3pt}#4}}\vspace{-\baselineskip}
    }
    

    
    % Prevent overflowing lines due to hard-to-break entities
    \sloppy 
    % Setup hyperref package
    \hypersetup{
      breaklinks=true,  % so long urls are correctly broken across lines
      colorlinks=true,
      urlcolor=urlcolor,
      linkcolor=linkcolor,
      citecolor=citecolor,
      }
    % Slightly bigger margins than the latex defaults
    
    \geometry{verbose,tmargin=1in,bmargin=1in,lmargin=1in,rmargin=1in}
    
    

\begin{document}
    
    \maketitle
    
    

    
    \hypertarget{practice-lab-deep-learning-for-content-based-filtering}{%
\section{\texorpdfstring{ Practice lab: Deep Learning for Content-Based
Filtering}{ Practice lab: Deep Learning for Content-Based Filtering}}\label{practice-lab-deep-learning-for-content-based-filtering}}

In this exercise, you will implement content-based filtering using a
neural network to build a recommender system for movies.

\hypertarget{outline}{%
\section{Outline}\label{outline}}

\begin{itemize}
\tightlist
\item
  Section \ref{1}
\item
  Section \ref{2}
\item
  Section \ref{3}

  \begin{itemize}
  \tightlist
  \item
    Section \ref{31}
  \item
    Section \ref{32}
  \end{itemize}
\item
  Section \ref{4}

  \begin{itemize}
  \tightlist
  \item
    Section \ref{ex01}
  \end{itemize}
\item
  Section \ref{5}

  \begin{itemize}
  \tightlist
  \item
    Section \ref{51}
  \item
    Section \ref{52}
  \item
    Section \ref{53}

    \begin{itemize}
    \tightlist
    \item
      Section \ref{ex02}
    \end{itemize}
  \end{itemize}
\item
  Section \ref{6}
\end{itemize}

    \emph{\textbf{NOTE:} To prevent errors from the autograder, you are not
allowed to edit or delete non-graded cells in this lab. Please also
refrain from adding any new cells. \textbf{Once you have passed this
assignment} and want to experiment with any of the non-graded code, you
may follow the instructions at the bottom of this notebook.}

    \#\# 1 - Packages We will use familiar packages, NumPy, TensorFlow and
helpful routines from
\href{https://scikit-learn.org/stable/}{scikit-learn}. We will also use
\href{https://pypi.org/project/tabulate/}{tabulate} to neatly print
tables and \href{https://pandas.pydata.org/}{Pandas} to organize tabular
data.

    \begin{tcolorbox}[breakable, size=fbox, boxrule=1pt, pad at break*=1mm,colback=cellbackground, colframe=cellborder]
\prompt{In}{incolor}{1}{\boxspacing}
\begin{Verbatim}[commandchars=\\\{\}]
\PY{k+kn}{import} \PY{n+nn}{numpy} \PY{k}{as} \PY{n+nn}{np}
\PY{k+kn}{import} \PY{n+nn}{numpy}\PY{n+nn}{.}\PY{n+nn}{ma} \PY{k}{as} \PY{n+nn}{ma}
\PY{k+kn}{import} \PY{n+nn}{pandas} \PY{k}{as} \PY{n+nn}{pd}
\PY{k+kn}{import} \PY{n+nn}{tensorflow} \PY{k}{as} \PY{n+nn}{tf}
\PY{k+kn}{from} \PY{n+nn}{tensorflow} \PY{k+kn}{import} \PY{n}{keras}
\PY{k+kn}{from} \PY{n+nn}{sklearn}\PY{n+nn}{.}\PY{n+nn}{preprocessing} \PY{k+kn}{import} \PY{n}{StandardScaler}\PY{p}{,} \PY{n}{MinMaxScaler}
\PY{k+kn}{from} \PY{n+nn}{sklearn}\PY{n+nn}{.}\PY{n+nn}{model\PYZus{}selection} \PY{k+kn}{import} \PY{n}{train\PYZus{}test\PYZus{}split}
\PY{k+kn}{import} \PY{n+nn}{tabulate}
\PY{k+kn}{from} \PY{n+nn}{recsysNN\PYZus{}utils} \PY{k+kn}{import} \PY{o}{*}
\PY{n}{pd}\PY{o}{.}\PY{n}{set\PYZus{}option}\PY{p}{(}\PY{l+s+s2}{\PYZdq{}}\PY{l+s+s2}{display.precision}\PY{l+s+s2}{\PYZdq{}}\PY{p}{,} \PY{l+m+mi}{1}\PY{p}{)}
\end{Verbatim}
\end{tcolorbox}

    \#\# 2 - Movie ratings dataset The data set is derived from the
\href{https://grouplens.org/datasets/movielens/latest/}{MovieLens
ml-latest-small} dataset.

{[}F. Maxwell Harper and Joseph A. Konstan. 2015. The MovieLens
Datasets: History and Context. ACM Transactions on Interactive
Intelligent Systems (TiiS) 5, 4: 19:1--19:19.
\url{https://doi.org/10.1145/2827872}{]}

The original dataset has roughly 9000 movies rated by 600 users with
ratings on a scale of 0.5 to 5 in 0.5 step increments. The dataset has
been reduced in size to focus on movies from the years since 2000 and
popular genres. The reduced dataset has \(n_u = 397\) users,
\(n_m= 847\) movies and 25521 ratings. For each movie, the dataset
provides a movie title, release date, and one or more genres. For
example ``Toy Story 3'' was released in 2010 and has several genres:
``Adventure\textbar Animation\textbar Children\textbar Comedy\textbar Fantasy''.
This dataset contains little information about users other than their
ratings. This dataset is used to create training vectors for the neural
networks described below. Let's learn a bit more about this data set.
The table below shows the top 10 movies ranked by the number of ratings.
These movies also happen to have high average ratings. How many of these
movies have you watched?

    \begin{tcolorbox}[breakable, size=fbox, boxrule=1pt, pad at break*=1mm,colback=cellbackground, colframe=cellborder]
\prompt{In}{incolor}{2}{\boxspacing}
\begin{Verbatim}[commandchars=\\\{\}]
\PY{n}{top10\PYZus{}df} \PY{o}{=} \PY{n}{pd}\PY{o}{.}\PY{n}{read\PYZus{}csv}\PY{p}{(}\PY{l+s+s2}{\PYZdq{}}\PY{l+s+s2}{./data/content\PYZus{}top10\PYZus{}df.csv}\PY{l+s+s2}{\PYZdq{}}\PY{p}{)}
\PY{n}{bygenre\PYZus{}df} \PY{o}{=} \PY{n}{pd}\PY{o}{.}\PY{n}{read\PYZus{}csv}\PY{p}{(}\PY{l+s+s2}{\PYZdq{}}\PY{l+s+s2}{./data/content\PYZus{}bygenre\PYZus{}df.csv}\PY{l+s+s2}{\PYZdq{}}\PY{p}{)}
\PY{n}{top10\PYZus{}df}
\end{Verbatim}
\end{tcolorbox}

            \begin{tcolorbox}[breakable, size=fbox, boxrule=.5pt, pad at break*=1mm, opacityfill=0]
\prompt{Out}{outcolor}{2}{\boxspacing}
\begin{Verbatim}[commandchars=\\\{\}]
   movie id  num ratings  ave rating  \textbackslash{}
0      4993          198         4.1
1      5952          188         4.0
2      7153          185         4.1
3      4306          170         3.9
4     58559          149         4.2
5      6539          149         3.8
6     79132          143         4.1
7      6377          141         4.0
8      4886          132         3.9
9      7361          131         4.2

                                               title  \textbackslash{}
0  Lord of the Rings: The Fellowship of the Ring,{\ldots}
1             Lord of the Rings: The Two Towers, The
2     Lord of the Rings: The Return of the King, The
3                                              Shrek
4                                   Dark Knight, The
5  Pirates of the Caribbean: The Curse of the Bla{\ldots}
6                                          Inception
7                                       Finding Nemo
8                                     Monsters, Inc.
9              Eternal Sunshine of the Spotless Mind

                                              genres
0                                  Adventure|Fantasy
1                                  Adventure|Fantasy
2                     Action|Adventure|Drama|Fantasy
3  Adventure|Animation|Children|Comedy|Fantasy|Ro{\ldots}
4                                 Action|Crime|Drama
5                    Action|Adventure|Comedy|Fantasy
6         Action|Crime|Drama|Mystery|Sci-Fi|Thriller
7                Adventure|Animation|Children|Comedy
8        Adventure|Animation|Children|Comedy|Fantasy
9                               Drama|Romance|Sci-Fi
\end{Verbatim}
\end{tcolorbox}
        
    The next table shows information sorted by genre. The number of ratings
per genre vary substantially. Note that a movie may have multiple
genre's so the sum of the ratings below is larger than the number of
original ratings.

    \begin{tcolorbox}[breakable, size=fbox, boxrule=1pt, pad at break*=1mm,colback=cellbackground, colframe=cellborder]
\prompt{In}{incolor}{3}{\boxspacing}
\begin{Verbatim}[commandchars=\\\{\}]
\PY{n}{bygenre\PYZus{}df}
\end{Verbatim}
\end{tcolorbox}

            \begin{tcolorbox}[breakable, size=fbox, boxrule=.5pt, pad at break*=1mm, opacityfill=0]
\prompt{Out}{outcolor}{3}{\boxspacing}
\begin{Verbatim}[commandchars=\\\{\}]
          genre  num movies  ave rating/genre  ratings per genre
0        Action         321               3.4              10377
1     Adventure         234               3.4               8785
2     Animation          76               3.6               2588
3      Children          69               3.4               2472
4        Comedy         326               3.4               8911
5         Crime         139               3.5               4671
6   Documentary          13               3.8                280
7         Drama         342               3.6              10201
8       Fantasy         124               3.4               4468
9        Horror          56               3.2               1345
10      Mystery          68               3.6               2497
11      Romance         151               3.4               4468
12       Sci-Fi         174               3.4               5894
13     Thriller         245               3.4               7659
\end{Verbatim}
\end{tcolorbox}
        
    \#\# 3 - Content-based filtering with a neural network

In the collaborative filtering lab, you generated two vectors, a user
vector and an item/movie vector whose dot product would predict a
rating. The vectors were derived solely from the ratings.

Content-based filtering also generates a user and movie feature vector
but recognizes there may be other information available about the user
and/or movie that may improve the prediction. The additional information
is provided to a neural network which then generates the user and movie
vector as shown below.

\#\#\# 3.1 Training Data The movie content provided to the network is a
combination of the original data and some `engineered features'. Recall
the feature engineering discussion and lab from Course 1, Week 2, lab 4.
The original features are the year the movie was released and the
movie's genre's presented as a one-hot vector. There are 14 genres. The
engineered feature is an average rating derived from the user ratings.

The user content is composed of engineered features. A per genre average
rating is computed per user. Additionally, a user id, rating count and
rating average are available but not included in the training or
prediction content. They are carried with the data set because they are
useful in interpreting data.

The training set consists of all the ratings made by the users in the
data set. Some ratings are repeated to boost the number of training
examples of underrepresented genre's. The training set is split into two
arrays with the same number of entries, a user array and a movie/item
array.

Below, let's load and display some of the data.

    \begin{tcolorbox}[breakable, size=fbox, boxrule=1pt, pad at break*=1mm,colback=cellbackground, colframe=cellborder]
\prompt{In}{incolor}{4}{\boxspacing}
\begin{Verbatim}[commandchars=\\\{\}]
\PY{c+c1}{\PYZsh{} Load Data, set configuration variables}
\PY{n}{item\PYZus{}train}\PY{p}{,} \PY{n}{user\PYZus{}train}\PY{p}{,} \PY{n}{y\PYZus{}train}\PY{p}{,} \PY{n}{item\PYZus{}features}\PY{p}{,} \PY{n}{user\PYZus{}features}\PY{p}{,} \PY{n}{item\PYZus{}vecs}\PY{p}{,} \PY{n}{movie\PYZus{}dict}\PY{p}{,} \PY{n}{user\PYZus{}to\PYZus{}genre} \PY{o}{=} \PY{n}{load\PYZus{}data}\PY{p}{(}\PY{p}{)}

\PY{n}{num\PYZus{}user\PYZus{}features} \PY{o}{=} \PY{n}{user\PYZus{}train}\PY{o}{.}\PY{n}{shape}\PY{p}{[}\PY{l+m+mi}{1}\PY{p}{]} \PY{o}{\PYZhy{}} \PY{l+m+mi}{3}  \PY{c+c1}{\PYZsh{} remove userid, rating count and ave rating during training}
\PY{n}{num\PYZus{}item\PYZus{}features} \PY{o}{=} \PY{n}{item\PYZus{}train}\PY{o}{.}\PY{n}{shape}\PY{p}{[}\PY{l+m+mi}{1}\PY{p}{]} \PY{o}{\PYZhy{}} \PY{l+m+mi}{1}  \PY{c+c1}{\PYZsh{} remove movie id at train time}
\PY{n}{uvs} \PY{o}{=} \PY{l+m+mi}{3}  \PY{c+c1}{\PYZsh{} user genre vector start}
\PY{n}{ivs} \PY{o}{=} \PY{l+m+mi}{3}  \PY{c+c1}{\PYZsh{} item genre vector start}
\PY{n}{u\PYZus{}s} \PY{o}{=} \PY{l+m+mi}{3}  \PY{c+c1}{\PYZsh{} start of columns to use in training, user}
\PY{n}{i\PYZus{}s} \PY{o}{=} \PY{l+m+mi}{1}  \PY{c+c1}{\PYZsh{} start of columns to use in training, items}
\PY{n+nb}{print}\PY{p}{(}\PY{l+s+sa}{f}\PY{l+s+s2}{\PYZdq{}}\PY{l+s+s2}{Number of training vectors: }\PY{l+s+si}{\PYZob{}}\PY{n+nb}{len}\PY{p}{(}\PY{n}{item\PYZus{}train}\PY{p}{)}\PY{l+s+si}{\PYZcb{}}\PY{l+s+s2}{\PYZdq{}}\PY{p}{)}
\end{Verbatim}
\end{tcolorbox}

    \begin{Verbatim}[commandchars=\\\{\}]
Number of training vectors: 50884
    \end{Verbatim}

    Let's look at the first few entries in the user training array.

    \begin{tcolorbox}[breakable, size=fbox, boxrule=1pt, pad at break*=1mm,colback=cellbackground, colframe=cellborder]
\prompt{In}{incolor}{5}{\boxspacing}
\begin{Verbatim}[commandchars=\\\{\}]
\PY{n}{pprint\PYZus{}train}\PY{p}{(}\PY{n}{user\PYZus{}train}\PY{p}{,} \PY{n}{user\PYZus{}features}\PY{p}{,} \PY{n}{uvs}\PY{p}{,}  \PY{n}{u\PYZus{}s}\PY{p}{,} \PY{n}{maxcount}\PY{o}{=}\PY{l+m+mi}{5}\PY{p}{)}
\end{Verbatim}
\end{tcolorbox}

            \begin{tcolorbox}[breakable, size=fbox, boxrule=.5pt, pad at break*=1mm, opacityfill=0]
\prompt{Out}{outcolor}{5}{\boxspacing}
\begin{Verbatim}[commandchars=\\\{\}]
'<table>\textbackslash{}n<thead>\textbackslash{}n<tr><th style="text-align: center;"> [user id] </th><th
style="text-align: center;"> [rating count] </th><th style="text-align:
center;"> [rating ave] </th><th style="text-align: center;"> Act ion </th><th
style="text-align: center;"> Adve nture </th><th style="text-align: center;">
Anim ation </th><th style="text-align: center;"> Chil dren </th><th style="text-
align: center;"> Com edy </th><th style="text-align: center;"> Crime </th><th
style="text-align: center;"> Docum entary </th><th style="text-align: center;">
Drama </th><th style="text-align: center;"> Fan tasy </th><th style="text-align:
center;"> Hor ror </th><th style="text-align: center;"> Mys tery </th><th
style="text-align: center;"> Rom ance </th><th style="text-align: center;"> Sci
-Fi </th><th style="text-align: center;"> Thri ller
</th></tr>\textbackslash{}n</thead>\textbackslash{}n<tbody>\textbackslash{}n<tr><td style="text-align: center;">     2
</td><td style="text-align: center;">       22       </td><td style="text-align:
center;">     4.0      </td><td style="text-align: center;">   4.0   </td><td
style="text-align: center;">    4.2     </td><td style="text-align: center;">
0.0     </td><td style="text-align: center;">    0.0    </td><td style="text-
align: center;">   4.0   </td><td style="text-align: center;">  4.1  </td><td
style="text-align: center;">     4.0      </td><td style="text-align: center;">
4.0  </td><td style="text-align: center;">   0.0    </td><td style="text-align:
center;">   3.0   </td><td style="text-align: center;">   4.0    </td><td
style="text-align: center;">   0.0    </td><td style="text-align: center;">
3.9   </td><td style="text-align: center;">    3.9    </td></tr>\textbackslash{}n<tr><td
style="text-align: center;">     2     </td><td style="text-align: center;">
22       </td><td style="text-align: center;">     4.0      </td><td
style="text-align: center;">   4.0   </td><td style="text-align: center;">
4.2     </td><td style="text-align: center;">    0.0     </td><td style="text-
align: center;">    0.0    </td><td style="text-align: center;">   4.0
</td><td style="text-align: center;">  4.1  </td><td style="text-align:
center;">     4.0      </td><td style="text-align: center;">  4.0  </td><td
style="text-align: center;">   0.0    </td><td style="text-align: center;">
3.0   </td><td style="text-align: center;">   4.0    </td><td style="text-align:
center;">   0.0    </td><td style="text-align: center;">   3.9   </td><td
style="text-align: center;">    3.9    </td></tr>\textbackslash{}n<tr><td style="text-align:
center;">     2     </td><td style="text-align: center;">       22
</td><td style="text-align: center;">     4.0      </td><td style="text-align:
center;">   4.0   </td><td style="text-align: center;">    4.2     </td><td
style="text-align: center;">    0.0     </td><td style="text-align: center;">
0.0    </td><td style="text-align: center;">   4.0   </td><td style="text-align:
center;">  4.1  </td><td style="text-align: center;">     4.0      </td><td
style="text-align: center;">  4.0  </td><td style="text-align: center;">   0.0
</td><td style="text-align: center;">   3.0   </td><td style="text-align:
center;">   4.0    </td><td style="text-align: center;">   0.0    </td><td
style="text-align: center;">   3.9   </td><td style="text-align: center;">
3.9    </td></tr>\textbackslash{}n<tr><td style="text-align: center;">     2     </td><td
style="text-align: center;">       22       </td><td style="text-align:
center;">     4.0      </td><td style="text-align: center;">   4.0   </td><td
style="text-align: center;">    4.2     </td><td style="text-align: center;">
0.0     </td><td style="text-align: center;">    0.0    </td><td style="text-
align: center;">   4.0   </td><td style="text-align: center;">  4.1  </td><td
style="text-align: center;">     4.0      </td><td style="text-align: center;">
4.0  </td><td style="text-align: center;">   0.0    </td><td style="text-align:
center;">   3.0   </td><td style="text-align: center;">   4.0    </td><td
style="text-align: center;">   0.0    </td><td style="text-align: center;">
3.9   </td><td style="text-align: center;">    3.9    </td></tr>\textbackslash{}n<tr><td
style="text-align: center;">     2     </td><td style="text-align: center;">
22       </td><td style="text-align: center;">     4.0      </td><td
style="text-align: center;">   4.0   </td><td style="text-align: center;">
4.2     </td><td style="text-align: center;">    0.0     </td><td style="text-
align: center;">    0.0    </td><td style="text-align: center;">   4.0
</td><td style="text-align: center;">  4.1  </td><td style="text-align:
center;">     4.0      </td><td style="text-align: center;">  4.0  </td><td
style="text-align: center;">   0.0    </td><td style="text-align: center;">
3.0   </td><td style="text-align: center;">   4.0    </td><td style="text-align:
center;">   0.0    </td><td style="text-align: center;">   3.9   </td><td
style="text-align: center;">    3.9    </td></tr>\textbackslash{}n</tbody>\textbackslash{}n</table>'
\end{Verbatim}
\end{tcolorbox}
        
    Some of the user and item/movie features are not used in training. In
the table above, the features in brackets ``{[}{]}'' such as the ``user
id'', ``rating count'' and ``rating ave'' are not included when the
model is trained and used. Above you can see the per genre rating
average for user 2. Zero entries are genre's which the user had not
rated. The user vector is the same for all the movies rated by a user.\\
Let's look at the first few entries of the movie/item array.

    \begin{tcolorbox}[breakable, size=fbox, boxrule=1pt, pad at break*=1mm,colback=cellbackground, colframe=cellborder]
\prompt{In}{incolor}{6}{\boxspacing}
\begin{Verbatim}[commandchars=\\\{\}]
\PY{n}{pprint\PYZus{}train}\PY{p}{(}\PY{n}{item\PYZus{}train}\PY{p}{,} \PY{n}{item\PYZus{}features}\PY{p}{,} \PY{n}{ivs}\PY{p}{,} \PY{n}{i\PYZus{}s}\PY{p}{,} \PY{n}{maxcount}\PY{o}{=}\PY{l+m+mi}{5}\PY{p}{,} \PY{n}{user}\PY{o}{=}\PY{k+kc}{False}\PY{p}{)}
\end{Verbatim}
\end{tcolorbox}

            \begin{tcolorbox}[breakable, size=fbox, boxrule=.5pt, pad at break*=1mm, opacityfill=0]
\prompt{Out}{outcolor}{6}{\boxspacing}
\begin{Verbatim}[commandchars=\\\{\}]
'<table>\textbackslash{}n<thead>\textbackslash{}n<tr><th style="text-align: center;"> [movie id] </th><th
style="text-align: center;"> year </th><th style="text-align: center;"> ave
rating </th><th style="text-align: center;"> Act ion </th><th style="text-align:
center;"> Adve nture </th><th style="text-align: center;"> Anim ation </th><th
style="text-align: center;"> Chil dren </th><th style="text-align: center;"> Com
edy </th><th style="text-align: center;"> Crime </th><th style="text-align:
center;"> Docum entary </th><th style="text-align: center;"> Drama </th><th
style="text-align: center;"> Fan tasy </th><th style="text-align: center;"> Hor
ror </th><th style="text-align: center;"> Mys tery </th><th style="text-align:
center;"> Rom ance </th><th style="text-align: center;"> Sci -Fi </th><th
style="text-align: center;"> Thri ller </th></tr>\textbackslash{}n</thead>\textbackslash{}n<tbody>\textbackslash{}n<tr><td
style="text-align: center;">    6874    </td><td style="text-align: center;">
2003 </td><td style="text-align: center;">    4.0     </td><td style="text-
align: center;">    1    </td><td style="text-align: center;">     0
</td><td style="text-align: center;">     0      </td><td style="text-align:
center;">     0     </td><td style="text-align: center;">    0    </td><td
style="text-align: center;">   1   </td><td style="text-align: center;">      0
</td><td style="text-align: center;">   0   </td><td style="text-align:
center;">    0     </td><td style="text-align: center;">    0    </td><td
style="text-align: center;">    0     </td><td style="text-align: center;">    0
</td><td style="text-align: center;">    0    </td><td style="text-align:
center;">     1     </td></tr>\textbackslash{}n<tr><td style="text-align: center;">    8798
</td><td style="text-align: center;"> 2004 </td><td style="text-align: center;">
3.8     </td><td style="text-align: center;">    1    </td><td style="text-
align: center;">     0      </td><td style="text-align: center;">     0
</td><td style="text-align: center;">     0     </td><td style="text-align:
center;">    0    </td><td style="text-align: center;">   1   </td><td
style="text-align: center;">      0       </td><td style="text-align: center;">
1   </td><td style="text-align: center;">    0     </td><td style="text-align:
center;">    0    </td><td style="text-align: center;">    0     </td><td
style="text-align: center;">    0     </td><td style="text-align: center;">    0
</td><td style="text-align: center;">     1     </td></tr>\textbackslash{}n<tr><td style="text-
align: center;">   46970    </td><td style="text-align: center;"> 2006 </td><td
style="text-align: center;">    3.2     </td><td style="text-align: center;">
1    </td><td style="text-align: center;">     0      </td><td style="text-
align: center;">     0      </td><td style="text-align: center;">     0
</td><td style="text-align: center;">    1    </td><td style="text-align:
center;">   0   </td><td style="text-align: center;">      0       </td><td
style="text-align: center;">   0   </td><td style="text-align: center;">    0
</td><td style="text-align: center;">    0    </td><td style="text-align:
center;">    0     </td><td style="text-align: center;">    0     </td><td
style="text-align: center;">    0    </td><td style="text-align: center;">     0
</td></tr>\textbackslash{}n<tr><td style="text-align: center;">   48516    </td><td
style="text-align: center;"> 2006 </td><td style="text-align: center;">    4.3
</td><td style="text-align: center;">    0    </td><td style="text-align:
center;">     0      </td><td style="text-align: center;">     0      </td><td
style="text-align: center;">     0     </td><td style="text-align: center;">
0    </td><td style="text-align: center;">   1   </td><td style="text-align:
center;">      0       </td><td style="text-align: center;">   1   </td><td
style="text-align: center;">    0     </td><td style="text-align: center;">    0
</td><td style="text-align: center;">    0     </td><td style="text-align:
center;">    0     </td><td style="text-align: center;">    0    </td><td
style="text-align: center;">     1     </td></tr>\textbackslash{}n<tr><td style="text-align:
center;">   58559    </td><td style="text-align: center;"> 2008 </td><td
style="text-align: center;">    4.2     </td><td style="text-align: center;">
1    </td><td style="text-align: center;">     0      </td><td style="text-
align: center;">     0      </td><td style="text-align: center;">     0
</td><td style="text-align: center;">    0    </td><td style="text-align:
center;">   1   </td><td style="text-align: center;">      0       </td><td
style="text-align: center;">   1   </td><td style="text-align: center;">    0
</td><td style="text-align: center;">    0    </td><td style="text-align:
center;">    0     </td><td style="text-align: center;">    0     </td><td
style="text-align: center;">    0    </td><td style="text-align: center;">     0
</td></tr>\textbackslash{}n</tbody>\textbackslash{}n</table>'
\end{Verbatim}
\end{tcolorbox}
        
    Above, the movie array contains the year the film was released, the
average rating and an indicator for each potential genre. The indicator
is one for each genre that applies to the movie. The movie id is not
used in training but is useful when interpreting the data.

    \begin{tcolorbox}[breakable, size=fbox, boxrule=1pt, pad at break*=1mm,colback=cellbackground, colframe=cellborder]
\prompt{In}{incolor}{7}{\boxspacing}
\begin{Verbatim}[commandchars=\\\{\}]
\PY{n+nb}{print}\PY{p}{(}\PY{l+s+sa}{f}\PY{l+s+s2}{\PYZdq{}}\PY{l+s+s2}{y\PYZus{}train[:5]: }\PY{l+s+si}{\PYZob{}}\PY{n}{y\PYZus{}train}\PY{p}{[}\PY{p}{:}\PY{l+m+mi}{5}\PY{p}{]}\PY{l+s+si}{\PYZcb{}}\PY{l+s+s2}{\PYZdq{}}\PY{p}{)}
\end{Verbatim}
\end{tcolorbox}

    \begin{Verbatim}[commandchars=\\\{\}]
y\_train[:5]: [4.  3.5 4.  4.  4.5]
    \end{Verbatim}

    The target, y, is the movie rating given by the user.

    Above, we can see that movie 6874 is an Action/Crime/Thriller movie
released in 2003. User 2 rates action movies as 3.9 on average.
MovieLens users gave the movie an average rating of 4. `y' is 4
indicating user 2 rated movie 6874 as a 4 as well. A single training
example consists of a row from both the user and item arrays and a
rating from y\_train.

    \#\#\# 3.2 Preparing the training data Recall in Course 1, Week 2, you
explored feature scaling as a means of improving convergence. We'll
scale the input features using the
\href{https://scikit-learn.org/stable/modules/generated/sklearn.preprocessing.StandardScaler.html}{scikit
learn StandardScaler}. This was used in Course 1, Week 2, Lab 5. Below,
the inverse\_transform is also shown to produce the original inputs.
We'll scale the target ratings using a Min Max Scaler which scales the
target to be between -1 and 1.
\href{https://scikit-learn.org/stable/modules/generated/sklearn.preprocessing.MinMaxScaler.html}{scikit
learn MinMaxScaler}

    \begin{tcolorbox}[breakable, size=fbox, boxrule=1pt, pad at break*=1mm,colback=cellbackground, colframe=cellborder]
\prompt{In}{incolor}{8}{\boxspacing}
\begin{Verbatim}[commandchars=\\\{\}]
\PY{c+c1}{\PYZsh{} scale training data}
\PY{n}{item\PYZus{}train\PYZus{}unscaled} \PY{o}{=} \PY{n}{item\PYZus{}train}
\PY{n}{user\PYZus{}train\PYZus{}unscaled} \PY{o}{=} \PY{n}{user\PYZus{}train}
\PY{n}{y\PYZus{}train\PYZus{}unscaled}    \PY{o}{=} \PY{n}{y\PYZus{}train}

\PY{n}{scalerItem} \PY{o}{=} \PY{n}{StandardScaler}\PY{p}{(}\PY{p}{)}
\PY{n}{scalerItem}\PY{o}{.}\PY{n}{fit}\PY{p}{(}\PY{n}{item\PYZus{}train}\PY{p}{)}
\PY{n}{item\PYZus{}train} \PY{o}{=} \PY{n}{scalerItem}\PY{o}{.}\PY{n}{transform}\PY{p}{(}\PY{n}{item\PYZus{}train}\PY{p}{)}

\PY{n}{scalerUser} \PY{o}{=} \PY{n}{StandardScaler}\PY{p}{(}\PY{p}{)}
\PY{n}{scalerUser}\PY{o}{.}\PY{n}{fit}\PY{p}{(}\PY{n}{user\PYZus{}train}\PY{p}{)}
\PY{n}{user\PYZus{}train} \PY{o}{=} \PY{n}{scalerUser}\PY{o}{.}\PY{n}{transform}\PY{p}{(}\PY{n}{user\PYZus{}train}\PY{p}{)}

\PY{n}{scalerTarget} \PY{o}{=} \PY{n}{MinMaxScaler}\PY{p}{(}\PY{p}{(}\PY{o}{\PYZhy{}}\PY{l+m+mi}{1}\PY{p}{,} \PY{l+m+mi}{1}\PY{p}{)}\PY{p}{)}
\PY{n}{scalerTarget}\PY{o}{.}\PY{n}{fit}\PY{p}{(}\PY{n}{y\PYZus{}train}\PY{o}{.}\PY{n}{reshape}\PY{p}{(}\PY{o}{\PYZhy{}}\PY{l+m+mi}{1}\PY{p}{,} \PY{l+m+mi}{1}\PY{p}{)}\PY{p}{)}
\PY{n}{y\PYZus{}train} \PY{o}{=} \PY{n}{scalerTarget}\PY{o}{.}\PY{n}{transform}\PY{p}{(}\PY{n}{y\PYZus{}train}\PY{o}{.}\PY{n}{reshape}\PY{p}{(}\PY{o}{\PYZhy{}}\PY{l+m+mi}{1}\PY{p}{,} \PY{l+m+mi}{1}\PY{p}{)}\PY{p}{)}
\PY{c+c1}{\PYZsh{}ynorm\PYZus{}test = scalerTarget.transform(y\PYZus{}test.reshape(\PYZhy{}1, 1))}

\PY{n+nb}{print}\PY{p}{(}\PY{n}{np}\PY{o}{.}\PY{n}{allclose}\PY{p}{(}\PY{n}{item\PYZus{}train\PYZus{}unscaled}\PY{p}{,} \PY{n}{scalerItem}\PY{o}{.}\PY{n}{inverse\PYZus{}transform}\PY{p}{(}\PY{n}{item\PYZus{}train}\PY{p}{)}\PY{p}{)}\PY{p}{)}
\PY{n+nb}{print}\PY{p}{(}\PY{n}{np}\PY{o}{.}\PY{n}{allclose}\PY{p}{(}\PY{n}{user\PYZus{}train\PYZus{}unscaled}\PY{p}{,} \PY{n}{scalerUser}\PY{o}{.}\PY{n}{inverse\PYZus{}transform}\PY{p}{(}\PY{n}{user\PYZus{}train}\PY{p}{)}\PY{p}{)}\PY{p}{)}
\end{Verbatim}
\end{tcolorbox}

    \begin{Verbatim}[commandchars=\\\{\}]
True
True
    \end{Verbatim}

    To allow us to evaluate the results, we will split the data into
training and test sets as was discussed in Course 2, Week 3. Here we
will use
\href{https://scikit-learn.org/stable/modules/generated/sklearn.model_selection.train_test_split.html}{sklean
train\_test\_split} to split and shuffle the data. Note that setting the
initial random state to the same value ensures item, user, and y are
shuffled identically.

    \begin{tcolorbox}[breakable, size=fbox, boxrule=1pt, pad at break*=1mm,colback=cellbackground, colframe=cellborder]
\prompt{In}{incolor}{9}{\boxspacing}
\begin{Verbatim}[commandchars=\\\{\}]
\PY{n}{item\PYZus{}train}\PY{p}{,} \PY{n}{item\PYZus{}test} \PY{o}{=} \PY{n}{train\PYZus{}test\PYZus{}split}\PY{p}{(}\PY{n}{item\PYZus{}train}\PY{p}{,} \PY{n}{train\PYZus{}size}\PY{o}{=}\PY{l+m+mf}{0.80}\PY{p}{,} \PY{n}{shuffle}\PY{o}{=}\PY{k+kc}{True}\PY{p}{,} \PY{n}{random\PYZus{}state}\PY{o}{=}\PY{l+m+mi}{1}\PY{p}{)}
\PY{n}{user\PYZus{}train}\PY{p}{,} \PY{n}{user\PYZus{}test} \PY{o}{=} \PY{n}{train\PYZus{}test\PYZus{}split}\PY{p}{(}\PY{n}{user\PYZus{}train}\PY{p}{,} \PY{n}{train\PYZus{}size}\PY{o}{=}\PY{l+m+mf}{0.80}\PY{p}{,} \PY{n}{shuffle}\PY{o}{=}\PY{k+kc}{True}\PY{p}{,} \PY{n}{random\PYZus{}state}\PY{o}{=}\PY{l+m+mi}{1}\PY{p}{)}
\PY{n}{y\PYZus{}train}\PY{p}{,} \PY{n}{y\PYZus{}test}       \PY{o}{=} \PY{n}{train\PYZus{}test\PYZus{}split}\PY{p}{(}\PY{n}{y\PYZus{}train}\PY{p}{,}    \PY{n}{train\PYZus{}size}\PY{o}{=}\PY{l+m+mf}{0.80}\PY{p}{,} \PY{n}{shuffle}\PY{o}{=}\PY{k+kc}{True}\PY{p}{,} \PY{n}{random\PYZus{}state}\PY{o}{=}\PY{l+m+mi}{1}\PY{p}{)}
\PY{n+nb}{print}\PY{p}{(}\PY{l+s+sa}{f}\PY{l+s+s2}{\PYZdq{}}\PY{l+s+s2}{movie/item training data shape: }\PY{l+s+si}{\PYZob{}}\PY{n}{item\PYZus{}train}\PY{o}{.}\PY{n}{shape}\PY{l+s+si}{\PYZcb{}}\PY{l+s+s2}{\PYZdq{}}\PY{p}{)}
\PY{n+nb}{print}\PY{p}{(}\PY{l+s+sa}{f}\PY{l+s+s2}{\PYZdq{}}\PY{l+s+s2}{movie/item test data shape: }\PY{l+s+si}{\PYZob{}}\PY{n}{item\PYZus{}test}\PY{o}{.}\PY{n}{shape}\PY{l+s+si}{\PYZcb{}}\PY{l+s+s2}{\PYZdq{}}\PY{p}{)}
\end{Verbatim}
\end{tcolorbox}

    \begin{Verbatim}[commandchars=\\\{\}]
movie/item training data shape: (40707, 17)
movie/item test data shape: (10177, 17)
    \end{Verbatim}

    The scaled, shuffled data now has a mean of zero.

    \begin{tcolorbox}[breakable, size=fbox, boxrule=1pt, pad at break*=1mm,colback=cellbackground, colframe=cellborder]
\prompt{In}{incolor}{10}{\boxspacing}
\begin{Verbatim}[commandchars=\\\{\}]
\PY{n}{pprint\PYZus{}train}\PY{p}{(}\PY{n}{user\PYZus{}train}\PY{p}{,} \PY{n}{user\PYZus{}features}\PY{p}{,} \PY{n}{uvs}\PY{p}{,} \PY{n}{u\PYZus{}s}\PY{p}{,} \PY{n}{maxcount}\PY{o}{=}\PY{l+m+mi}{5}\PY{p}{)}
\end{Verbatim}
\end{tcolorbox}

            \begin{tcolorbox}[breakable, size=fbox, boxrule=.5pt, pad at break*=1mm, opacityfill=0]
\prompt{Out}{outcolor}{10}{\boxspacing}
\begin{Verbatim}[commandchars=\\\{\}]
'<table>\textbackslash{}n<thead>\textbackslash{}n<tr><th style="text-align: center;"> [user id] </th><th
style="text-align: center;"> [rating count] </th><th style="text-align:
center;"> [rating ave] </th><th style="text-align: center;"> Act ion </th><th
style="text-align: center;"> Adve nture </th><th style="text-align: center;">
Anim ation </th><th style="text-align: center;"> Chil dren </th><th style="text-
align: center;"> Com edy </th><th style="text-align: center;"> Crime </th><th
style="text-align: center;"> Docum entary </th><th style="text-align: center;">
Drama </th><th style="text-align: center;"> Fan tasy </th><th style="text-align:
center;"> Hor ror </th><th style="text-align: center;"> Mys tery </th><th
style="text-align: center;"> Rom ance </th><th style="text-align: center;"> Sci
-Fi </th><th style="text-align: center;"> Thri ller
</th></tr>\textbackslash{}n</thead>\textbackslash{}n<tbody>\textbackslash{}n<tr><td style="text-align: center;">     1
</td><td style="text-align: center;">       0        </td><td style="text-align:
center;">     -1.0     </td><td style="text-align: center;">  -0.8   </td><td
style="text-align: center;">    -0.7    </td><td style="text-align: center;">
0.1     </td><td style="text-align: center;">   -0.0    </td><td style="text-
align: center;">  -1.2   </td><td style="text-align: center;"> -0.4  </td><td
style="text-align: center;">     0.6      </td><td style="text-align: center;">
-0.5  </td><td style="text-align: center;">   -0.5   </td><td style="text-align:
center;">  -0.1   </td><td style="text-align: center;">   -0.6   </td><td
style="text-align: center;">   -0.6   </td><td style="text-align: center;">
-0.7   </td><td style="text-align: center;">   -0.7    </td></tr>\textbackslash{}n<tr><td
style="text-align: center;">     0     </td><td style="text-align: center;">
1        </td><td style="text-align: center;">     -0.7     </td><td
style="text-align: center;">  -0.5   </td><td style="text-align: center;">
-0.7    </td><td style="text-align: center;">    -0.1    </td><td style="text-
align: center;">   -0.2    </td><td style="text-align: center;">  -0.6
</td><td style="text-align: center;"> -0.2  </td><td style="text-align:
center;">     0.7      </td><td style="text-align: center;"> -0.5  </td><td
style="text-align: center;">   -0.8   </td><td style="text-align: center;">
0.1   </td><td style="text-align: center;">   -0.0   </td><td style="text-align:
center;">   -0.6   </td><td style="text-align: center;">  -0.5   </td><td
style="text-align: center;">   -0.4    </td></tr>\textbackslash{}n<tr><td style="text-align:
center;">    -1     </td><td style="text-align: center;">       -1
</td><td style="text-align: center;">     -0.2     </td><td style="text-align:
center;">   0.3   </td><td style="text-align: center;">    -0.4    </td><td
style="text-align: center;">    0.4     </td><td style="text-align: center;">
0.5    </td><td style="text-align: center;">   1.0   </td><td style="text-align:
center;">  0.6  </td><td style="text-align: center;">     -1.2     </td><td
style="text-align: center;"> -0.3  </td><td style="text-align: center;">   -0.6
</td><td style="text-align: center;">  -2.3   </td><td style="text-align:
center;">   -0.1   </td><td style="text-align: center;">   0.0    </td><td
style="text-align: center;">   0.4   </td><td style="text-align: center;">
-0.0    </td></tr>\textbackslash{}n<tr><td style="text-align: center;">     0     </td><td
style="text-align: center;">       -1       </td><td style="text-align:
center;">     0.6      </td><td style="text-align: center;">   0.5   </td><td
style="text-align: center;">    0.5     </td><td style="text-align: center;">
0.2     </td><td style="text-align: center;">    0.6    </td><td style="text-
align: center;">  -0.1   </td><td style="text-align: center;">  0.5  </td><td
style="text-align: center;">     -1.2     </td><td style="text-align: center;">
0.9  </td><td style="text-align: center;">   1.2    </td><td style="text-align:
center;">  -2.3   </td><td style="text-align: center;">   -0.1   </td><td
style="text-align: center;">   0.0    </td><td style="text-align: center;">
0.2   </td><td style="text-align: center;">    0.3    </td></tr>\textbackslash{}n<tr><td
style="text-align: center;">    -1     </td><td style="text-align: center;">
0        </td><td style="text-align: center;">     0.7      </td><td
style="text-align: center;">   0.6   </td><td style="text-align: center;">
0.5     </td><td style="text-align: center;">    0.3     </td><td style="text-
align: center;">    0.5    </td><td style="text-align: center;">   0.4
</td><td style="text-align: center;">  0.6  </td><td style="text-align:
center;">     1.0      </td><td style="text-align: center;">  0.6  </td><td
style="text-align: center;">   0.3    </td><td style="text-align: center;">
0.8   </td><td style="text-align: center;">   0.8    </td><td style="text-align:
center;">   0.4    </td><td style="text-align: center;">   0.7   </td><td
style="text-align: center;">    0.7    </td></tr>\textbackslash{}n</tbody>\textbackslash{}n</table>'
\end{Verbatim}
\end{tcolorbox}
        
    \#\# 4 - Neural Network for content-based filtering Now, let's construct
a neural network as described in the figure above. It will have two
networks that are combined by a dot product. You will construct the two
networks. In this example, they will be identical. Note that these
networks do not need to be the same. If the user content was
substantially larger than the movie content, you might elect to increase
the complexity of the user network relative to the movie network. In
this case, the content is similar, so the networks are the same.

\#\#\# Exercise 1

\begin{itemize}
\tightlist
\item
  Use a Keras sequential model

  \begin{itemize}
  \tightlist
  \item
    The first layer is a dense layer with 256 units and a relu
    activation.
  \item
    The second layer is a dense layer with 128 units and a relu
    activation.
  \item
    The third layer is a dense layer with \texttt{num\_outputs} units
    and a linear or no activation.
  \end{itemize}
\end{itemize}

The remainder of the network will be provided. The provided code does
not use the Keras sequential model but instead uses the Keras
\href{https://keras.io/guides/functional_api/}{functional api}. This
format allows for more flexibility in how components are interconnected.

    \begin{tcolorbox}[breakable, size=fbox, boxrule=1pt, pad at break*=1mm,colback=cellbackground, colframe=cellborder]
\prompt{In}{incolor}{11}{\boxspacing}
\begin{Verbatim}[commandchars=\\\{\}]
\PY{c+c1}{\PYZsh{} GRADED\PYZus{}CELL}
\PY{c+c1}{\PYZsh{} UNQ\PYZus{}C1}

\PY{n}{num\PYZus{}outputs} \PY{o}{=} \PY{l+m+mi}{32}
\PY{n}{tf}\PY{o}{.}\PY{n}{random}\PY{o}{.}\PY{n}{set\PYZus{}seed}\PY{p}{(}\PY{l+m+mi}{1}\PY{p}{)}
\PY{n}{user\PYZus{}NN} \PY{o}{=} \PY{n}{tf}\PY{o}{.}\PY{n}{keras}\PY{o}{.}\PY{n}{models}\PY{o}{.}\PY{n}{Sequential}\PY{p}{(}\PY{p}{[}
    \PY{c+c1}{\PYZsh{}\PYZsh{}\PYZsh{} START CODE HERE \PYZsh{}\PYZsh{}\PYZsh{}     }
    \PY{n}{tf}\PY{o}{.}\PY{n}{keras}\PY{o}{.}\PY{n}{layers}\PY{o}{.}\PY{n}{Dense}\PY{p}{(}\PY{l+m+mi}{256}\PY{p}{,} \PY{n}{activation}\PY{o}{=}\PY{l+s+s1}{\PYZsq{}}\PY{l+s+s1}{relu}\PY{l+s+s1}{\PYZsq{}}\PY{p}{)}\PY{p}{,}
    \PY{n}{tf}\PY{o}{.}\PY{n}{keras}\PY{o}{.}\PY{n}{layers}\PY{o}{.}\PY{n}{Dense}\PY{p}{(}\PY{l+m+mi}{128}\PY{p}{,} \PY{n}{activation}\PY{o}{=}\PY{l+s+s1}{\PYZsq{}}\PY{l+s+s1}{relu}\PY{l+s+s1}{\PYZsq{}}\PY{p}{)}\PY{p}{,}
    \PY{n}{tf}\PY{o}{.}\PY{n}{keras}\PY{o}{.}\PY{n}{layers}\PY{o}{.}\PY{n}{Dense}\PY{p}{(}\PY{n}{num\PYZus{}outputs}\PY{p}{,} \PY{n}{activation}\PY{o}{=}\PY{l+s+s1}{\PYZsq{}}\PY{l+s+s1}{linear}\PY{l+s+s1}{\PYZsq{}}\PY{p}{)}\PY{p}{,}
  
    \PY{c+c1}{\PYZsh{}\PYZsh{}\PYZsh{} END CODE HERE \PYZsh{}\PYZsh{}\PYZsh{}  }
\PY{p}{]}\PY{p}{)}

\PY{n}{item\PYZus{}NN} \PY{o}{=} \PY{n}{tf}\PY{o}{.}\PY{n}{keras}\PY{o}{.}\PY{n}{models}\PY{o}{.}\PY{n}{Sequential}\PY{p}{(}\PY{p}{[}
    \PY{c+c1}{\PYZsh{}\PYZsh{}\PYZsh{} START CODE HERE \PYZsh{}\PYZsh{}\PYZsh{}     }
      \PY{n}{tf}\PY{o}{.}\PY{n}{keras}\PY{o}{.}\PY{n}{layers}\PY{o}{.}\PY{n}{Dense}\PY{p}{(}\PY{l+m+mi}{256}\PY{p}{,} \PY{n}{activation}\PY{o}{=}\PY{l+s+s1}{\PYZsq{}}\PY{l+s+s1}{relu}\PY{l+s+s1}{\PYZsq{}}\PY{p}{)}\PY{p}{,}
    \PY{n}{tf}\PY{o}{.}\PY{n}{keras}\PY{o}{.}\PY{n}{layers}\PY{o}{.}\PY{n}{Dense}\PY{p}{(}\PY{l+m+mi}{128}\PY{p}{,} \PY{n}{activation}\PY{o}{=}\PY{l+s+s1}{\PYZsq{}}\PY{l+s+s1}{relu}\PY{l+s+s1}{\PYZsq{}}\PY{p}{)}\PY{p}{,}
    \PY{n}{tf}\PY{o}{.}\PY{n}{keras}\PY{o}{.}\PY{n}{layers}\PY{o}{.}\PY{n}{Dense}\PY{p}{(}\PY{n}{num\PYZus{}outputs}\PY{p}{,} \PY{n}{activation}\PY{o}{=}\PY{l+s+s1}{\PYZsq{}}\PY{l+s+s1}{linear}\PY{l+s+s1}{\PYZsq{}}\PY{p}{)}\PY{p}{,}
  
  
    \PY{c+c1}{\PYZsh{}\PYZsh{}\PYZsh{} END CODE HERE \PYZsh{}\PYZsh{}\PYZsh{}  }
\PY{p}{]}\PY{p}{)}

\PY{c+c1}{\PYZsh{} create the user input and point to the base network}
\PY{n}{input\PYZus{}user} \PY{o}{=} \PY{n}{tf}\PY{o}{.}\PY{n}{keras}\PY{o}{.}\PY{n}{layers}\PY{o}{.}\PY{n}{Input}\PY{p}{(}\PY{n}{shape}\PY{o}{=}\PY{p}{(}\PY{n}{num\PYZus{}user\PYZus{}features}\PY{p}{)}\PY{p}{)}
\PY{n}{vu} \PY{o}{=} \PY{n}{user\PYZus{}NN}\PY{p}{(}\PY{n}{input\PYZus{}user}\PY{p}{)}
\PY{n}{vu} \PY{o}{=} \PY{n}{tf}\PY{o}{.}\PY{n}{linalg}\PY{o}{.}\PY{n}{l2\PYZus{}normalize}\PY{p}{(}\PY{n}{vu}\PY{p}{,} \PY{n}{axis}\PY{o}{=}\PY{l+m+mi}{1}\PY{p}{)}

\PY{c+c1}{\PYZsh{} create the item input and point to the base network}
\PY{n}{input\PYZus{}item} \PY{o}{=} \PY{n}{tf}\PY{o}{.}\PY{n}{keras}\PY{o}{.}\PY{n}{layers}\PY{o}{.}\PY{n}{Input}\PY{p}{(}\PY{n}{shape}\PY{o}{=}\PY{p}{(}\PY{n}{num\PYZus{}item\PYZus{}features}\PY{p}{)}\PY{p}{)}
\PY{n}{vm} \PY{o}{=} \PY{n}{item\PYZus{}NN}\PY{p}{(}\PY{n}{input\PYZus{}item}\PY{p}{)}
\PY{n}{vm} \PY{o}{=} \PY{n}{tf}\PY{o}{.}\PY{n}{linalg}\PY{o}{.}\PY{n}{l2\PYZus{}normalize}\PY{p}{(}\PY{n}{vm}\PY{p}{,} \PY{n}{axis}\PY{o}{=}\PY{l+m+mi}{1}\PY{p}{)}

\PY{c+c1}{\PYZsh{} compute the dot product of the two vectors vu and vm}
\PY{n}{output} \PY{o}{=} \PY{n}{tf}\PY{o}{.}\PY{n}{keras}\PY{o}{.}\PY{n}{layers}\PY{o}{.}\PY{n}{Dot}\PY{p}{(}\PY{n}{axes}\PY{o}{=}\PY{l+m+mi}{1}\PY{p}{)}\PY{p}{(}\PY{p}{[}\PY{n}{vu}\PY{p}{,} \PY{n}{vm}\PY{p}{]}\PY{p}{)}

\PY{c+c1}{\PYZsh{} specify the inputs and output of the model}
\PY{n}{model} \PY{o}{=} \PY{n}{tf}\PY{o}{.}\PY{n}{keras}\PY{o}{.}\PY{n}{Model}\PY{p}{(}\PY{p}{[}\PY{n}{input\PYZus{}user}\PY{p}{,} \PY{n}{input\PYZus{}item}\PY{p}{]}\PY{p}{,} \PY{n}{output}\PY{p}{)}

\PY{n}{model}\PY{o}{.}\PY{n}{summary}\PY{p}{(}\PY{p}{)}
\end{Verbatim}
\end{tcolorbox}

    \begin{Verbatim}[commandchars=\\\{\}]
Model: "model"
\_\_\_\_\_\_\_\_\_\_\_\_\_\_\_\_\_\_\_\_\_\_\_\_\_\_\_\_\_\_\_\_\_\_\_\_\_\_\_\_\_\_\_\_\_\_\_\_\_\_\_\_\_\_\_\_\_\_\_\_\_\_\_\_\_\_\_\_\_\_\_\_\_\_\_\_\_\_\_\_
\_\_\_\_\_\_\_\_\_\_\_\_\_\_\_\_\_\_
Layer (type)                    Output Shape         Param \#     Connected to
================================================================================
==================
input\_1 (InputLayer)            [(None, 14)]         0
\_\_\_\_\_\_\_\_\_\_\_\_\_\_\_\_\_\_\_\_\_\_\_\_\_\_\_\_\_\_\_\_\_\_\_\_\_\_\_\_\_\_\_\_\_\_\_\_\_\_\_\_\_\_\_\_\_\_\_\_\_\_\_\_\_\_\_\_\_\_\_\_\_\_\_\_\_\_\_\_
\_\_\_\_\_\_\_\_\_\_\_\_\_\_\_\_\_\_
input\_2 (InputLayer)            [(None, 16)]         0
\_\_\_\_\_\_\_\_\_\_\_\_\_\_\_\_\_\_\_\_\_\_\_\_\_\_\_\_\_\_\_\_\_\_\_\_\_\_\_\_\_\_\_\_\_\_\_\_\_\_\_\_\_\_\_\_\_\_\_\_\_\_\_\_\_\_\_\_\_\_\_\_\_\_\_\_\_\_\_\_
\_\_\_\_\_\_\_\_\_\_\_\_\_\_\_\_\_\_
sequential (Sequential)         (None, 32)           40864       input\_1[0][0]
\_\_\_\_\_\_\_\_\_\_\_\_\_\_\_\_\_\_\_\_\_\_\_\_\_\_\_\_\_\_\_\_\_\_\_\_\_\_\_\_\_\_\_\_\_\_\_\_\_\_\_\_\_\_\_\_\_\_\_\_\_\_\_\_\_\_\_\_\_\_\_\_\_\_\_\_\_\_\_\_
\_\_\_\_\_\_\_\_\_\_\_\_\_\_\_\_\_\_
sequential\_1 (Sequential)       (None, 32)           41376       input\_2[0][0]
\_\_\_\_\_\_\_\_\_\_\_\_\_\_\_\_\_\_\_\_\_\_\_\_\_\_\_\_\_\_\_\_\_\_\_\_\_\_\_\_\_\_\_\_\_\_\_\_\_\_\_\_\_\_\_\_\_\_\_\_\_\_\_\_\_\_\_\_\_\_\_\_\_\_\_\_\_\_\_\_
\_\_\_\_\_\_\_\_\_\_\_\_\_\_\_\_\_\_
tf\_op\_layer\_l2\_normalize/Square [(None, 32)]         0
sequential[0][0]
\_\_\_\_\_\_\_\_\_\_\_\_\_\_\_\_\_\_\_\_\_\_\_\_\_\_\_\_\_\_\_\_\_\_\_\_\_\_\_\_\_\_\_\_\_\_\_\_\_\_\_\_\_\_\_\_\_\_\_\_\_\_\_\_\_\_\_\_\_\_\_\_\_\_\_\_\_\_\_\_
\_\_\_\_\_\_\_\_\_\_\_\_\_\_\_\_\_\_
tf\_op\_layer\_l2\_normalize\_1/Squa [(None, 32)]         0
sequential\_1[0][0]
\_\_\_\_\_\_\_\_\_\_\_\_\_\_\_\_\_\_\_\_\_\_\_\_\_\_\_\_\_\_\_\_\_\_\_\_\_\_\_\_\_\_\_\_\_\_\_\_\_\_\_\_\_\_\_\_\_\_\_\_\_\_\_\_\_\_\_\_\_\_\_\_\_\_\_\_\_\_\_\_
\_\_\_\_\_\_\_\_\_\_\_\_\_\_\_\_\_\_
tf\_op\_layer\_l2\_normalize/Sum (T [(None, 1)]          0
tf\_op\_layer\_l2\_normalize/Square[0
\_\_\_\_\_\_\_\_\_\_\_\_\_\_\_\_\_\_\_\_\_\_\_\_\_\_\_\_\_\_\_\_\_\_\_\_\_\_\_\_\_\_\_\_\_\_\_\_\_\_\_\_\_\_\_\_\_\_\_\_\_\_\_\_\_\_\_\_\_\_\_\_\_\_\_\_\_\_\_\_
\_\_\_\_\_\_\_\_\_\_\_\_\_\_\_\_\_\_
tf\_op\_layer\_l2\_normalize\_1/Sum  [(None, 1)]          0
tf\_op\_layer\_l2\_normalize\_1/Square
\_\_\_\_\_\_\_\_\_\_\_\_\_\_\_\_\_\_\_\_\_\_\_\_\_\_\_\_\_\_\_\_\_\_\_\_\_\_\_\_\_\_\_\_\_\_\_\_\_\_\_\_\_\_\_\_\_\_\_\_\_\_\_\_\_\_\_\_\_\_\_\_\_\_\_\_\_\_\_\_
\_\_\_\_\_\_\_\_\_\_\_\_\_\_\_\_\_\_
tf\_op\_layer\_l2\_normalize/Maximu [(None, 1)]          0
tf\_op\_layer\_l2\_normalize/Sum[0][0
\_\_\_\_\_\_\_\_\_\_\_\_\_\_\_\_\_\_\_\_\_\_\_\_\_\_\_\_\_\_\_\_\_\_\_\_\_\_\_\_\_\_\_\_\_\_\_\_\_\_\_\_\_\_\_\_\_\_\_\_\_\_\_\_\_\_\_\_\_\_\_\_\_\_\_\_\_\_\_\_
\_\_\_\_\_\_\_\_\_\_\_\_\_\_\_\_\_\_
tf\_op\_layer\_l2\_normalize\_1/Maxi [(None, 1)]          0
tf\_op\_layer\_l2\_normalize\_1/Sum[0]
\_\_\_\_\_\_\_\_\_\_\_\_\_\_\_\_\_\_\_\_\_\_\_\_\_\_\_\_\_\_\_\_\_\_\_\_\_\_\_\_\_\_\_\_\_\_\_\_\_\_\_\_\_\_\_\_\_\_\_\_\_\_\_\_\_\_\_\_\_\_\_\_\_\_\_\_\_\_\_\_
\_\_\_\_\_\_\_\_\_\_\_\_\_\_\_\_\_\_
tf\_op\_layer\_l2\_normalize/Rsqrt  [(None, 1)]          0
tf\_op\_layer\_l2\_normalize/Maximum[
\_\_\_\_\_\_\_\_\_\_\_\_\_\_\_\_\_\_\_\_\_\_\_\_\_\_\_\_\_\_\_\_\_\_\_\_\_\_\_\_\_\_\_\_\_\_\_\_\_\_\_\_\_\_\_\_\_\_\_\_\_\_\_\_\_\_\_\_\_\_\_\_\_\_\_\_\_\_\_\_
\_\_\_\_\_\_\_\_\_\_\_\_\_\_\_\_\_\_
tf\_op\_layer\_l2\_normalize\_1/Rsqr [(None, 1)]          0
tf\_op\_layer\_l2\_normalize\_1/Maximu
\_\_\_\_\_\_\_\_\_\_\_\_\_\_\_\_\_\_\_\_\_\_\_\_\_\_\_\_\_\_\_\_\_\_\_\_\_\_\_\_\_\_\_\_\_\_\_\_\_\_\_\_\_\_\_\_\_\_\_\_\_\_\_\_\_\_\_\_\_\_\_\_\_\_\_\_\_\_\_\_
\_\_\_\_\_\_\_\_\_\_\_\_\_\_\_\_\_\_
tf\_op\_layer\_l2\_normalize (Tenso [(None, 32)]         0
sequential[0][0]
tf\_op\_layer\_l2\_normalize/Rsqrt[0]
\_\_\_\_\_\_\_\_\_\_\_\_\_\_\_\_\_\_\_\_\_\_\_\_\_\_\_\_\_\_\_\_\_\_\_\_\_\_\_\_\_\_\_\_\_\_\_\_\_\_\_\_\_\_\_\_\_\_\_\_\_\_\_\_\_\_\_\_\_\_\_\_\_\_\_\_\_\_\_\_
\_\_\_\_\_\_\_\_\_\_\_\_\_\_\_\_\_\_
tf\_op\_layer\_l2\_normalize\_1 (Ten [(None, 32)]         0
sequential\_1[0][0]
tf\_op\_layer\_l2\_normalize\_1/Rsqrt[
\_\_\_\_\_\_\_\_\_\_\_\_\_\_\_\_\_\_\_\_\_\_\_\_\_\_\_\_\_\_\_\_\_\_\_\_\_\_\_\_\_\_\_\_\_\_\_\_\_\_\_\_\_\_\_\_\_\_\_\_\_\_\_\_\_\_\_\_\_\_\_\_\_\_\_\_\_\_\_\_
\_\_\_\_\_\_\_\_\_\_\_\_\_\_\_\_\_\_
dot (Dot)                       (None, 1)            0
tf\_op\_layer\_l2\_normalize[0][0]
tf\_op\_layer\_l2\_normalize\_1[0][0]
================================================================================
==================
Total params: 82,240
Trainable params: 82,240
Non-trainable params: 0
\_\_\_\_\_\_\_\_\_\_\_\_\_\_\_\_\_\_\_\_\_\_\_\_\_\_\_\_\_\_\_\_\_\_\_\_\_\_\_\_\_\_\_\_\_\_\_\_\_\_\_\_\_\_\_\_\_\_\_\_\_\_\_\_\_\_\_\_\_\_\_\_\_\_\_\_\_\_\_\_
\_\_\_\_\_\_\_\_\_\_\_\_\_\_\_\_\_\_
    \end{Verbatim}

    \begin{tcolorbox}[breakable, size=fbox, boxrule=1pt, pad at break*=1mm,colback=cellbackground, colframe=cellborder]
\prompt{In}{incolor}{12}{\boxspacing}
\begin{Verbatim}[commandchars=\\\{\}]
\PY{c+c1}{\PYZsh{} Public tests}
\PY{k+kn}{from} \PY{n+nn}{public\PYZus{}tests} \PY{k+kn}{import} \PY{o}{*}
\PY{n}{test\PYZus{}tower}\PY{p}{(}\PY{n}{user\PYZus{}NN}\PY{p}{)}
\PY{n}{test\PYZus{}tower}\PY{p}{(}\PY{n}{item\PYZus{}NN}\PY{p}{)}
\end{Verbatim}
\end{tcolorbox}

    \begin{Verbatim}[commandchars=\\\{\}]
\textcolor{ansi-green-intense}{All tests passed!
}\textcolor{ansi-green-intense}{All tests passed!}
    \end{Verbatim}

    Click for hints

You can create a dense layer with a relu activation as shown.

\begin{Shaded}
\begin{Highlighting}[]
\NormalTok{user\_NN }\OperatorTok{=}\NormalTok{ tf.keras.models.Sequential([}
    \CommentTok{\#\#\# START CODE HERE }\AlertTok{\#\#\#}\CommentTok{     }
\NormalTok{  tf.keras.layers.Dense(}\DecValTok{256}\NormalTok{, activation}\OperatorTok{=}\StringTok{\textquotesingle{}relu\textquotesingle{}}\NormalTok{),}

    
    \CommentTok{\#\#\# }\RegionMarkerTok{END}\CommentTok{ CODE HERE }\AlertTok{\#\#\#}\CommentTok{  }
\NormalTok{])}

\NormalTok{item\_NN }\OperatorTok{=}\NormalTok{ tf.keras.models.Sequential([}
    \CommentTok{\#\#\# START CODE HERE }\AlertTok{\#\#\#}\CommentTok{     }
\NormalTok{  tf.keras.layers.Dense(}\DecValTok{256}\NormalTok{, activation}\OperatorTok{=}\StringTok{\textquotesingle{}relu\textquotesingle{}}\NormalTok{),}

    
    \CommentTok{\#\#\# }\RegionMarkerTok{END}\CommentTok{ CODE HERE }\AlertTok{\#\#\#}\CommentTok{  }
\NormalTok{])}
\end{Highlighting}
\end{Shaded}

Click for solution

\begin{Shaded}
\begin{Highlighting}[]
\NormalTok{user\_NN }\OperatorTok{=}\NormalTok{ tf.keras.models.Sequential([}
    \CommentTok{\#\#\# START CODE HERE }\AlertTok{\#\#\#}\CommentTok{     }
\NormalTok{  tf.keras.layers.Dense(}\DecValTok{256}\NormalTok{, activation}\OperatorTok{=}\StringTok{\textquotesingle{}relu\textquotesingle{}}\NormalTok{),}
\NormalTok{  tf.keras.layers.Dense(}\DecValTok{128}\NormalTok{, activation}\OperatorTok{=}\StringTok{\textquotesingle{}relu\textquotesingle{}}\NormalTok{),}
\NormalTok{  tf.keras.layers.Dense(num\_outputs),}
    \CommentTok{\#\#\# }\RegionMarkerTok{END}\CommentTok{ CODE HERE }\AlertTok{\#\#\#}\CommentTok{  }
\NormalTok{])}

\NormalTok{item\_NN }\OperatorTok{=}\NormalTok{ tf.keras.models.Sequential([}
    \CommentTok{\#\#\# START CODE HERE }\AlertTok{\#\#\#}\CommentTok{     }
\NormalTok{  tf.keras.layers.Dense(}\DecValTok{256}\NormalTok{, activation}\OperatorTok{=}\StringTok{\textquotesingle{}relu\textquotesingle{}}\NormalTok{),}
\NormalTok{  tf.keras.layers.Dense(}\DecValTok{128}\NormalTok{, activation}\OperatorTok{=}\StringTok{\textquotesingle{}relu\textquotesingle{}}\NormalTok{),}
\NormalTok{  tf.keras.layers.Dense(num\_outputs),}
    \CommentTok{\#\#\# }\RegionMarkerTok{END}\CommentTok{ CODE HERE }\AlertTok{\#\#\#}\CommentTok{  }
\NormalTok{])}
\end{Highlighting}
\end{Shaded}

    We will use a mean squared error loss and an Adam optimizer.

    \begin{tcolorbox}[breakable, size=fbox, boxrule=1pt, pad at break*=1mm,colback=cellbackground, colframe=cellborder]
\prompt{In}{incolor}{13}{\boxspacing}
\begin{Verbatim}[commandchars=\\\{\}]
\PY{n}{tf}\PY{o}{.}\PY{n}{random}\PY{o}{.}\PY{n}{set\PYZus{}seed}\PY{p}{(}\PY{l+m+mi}{1}\PY{p}{)}
\PY{n}{cost\PYZus{}fn} \PY{o}{=} \PY{n}{tf}\PY{o}{.}\PY{n}{keras}\PY{o}{.}\PY{n}{losses}\PY{o}{.}\PY{n}{MeanSquaredError}\PY{p}{(}\PY{p}{)}
\PY{n}{opt} \PY{o}{=} \PY{n}{keras}\PY{o}{.}\PY{n}{optimizers}\PY{o}{.}\PY{n}{Adam}\PY{p}{(}\PY{n}{learning\PYZus{}rate}\PY{o}{=}\PY{l+m+mf}{0.01}\PY{p}{)}
\PY{n}{model}\PY{o}{.}\PY{n}{compile}\PY{p}{(}\PY{n}{optimizer}\PY{o}{=}\PY{n}{opt}\PY{p}{,}
              \PY{n}{loss}\PY{o}{=}\PY{n}{cost\PYZus{}fn}\PY{p}{)}
\end{Verbatim}
\end{tcolorbox}

    \begin{tcolorbox}[breakable, size=fbox, boxrule=1pt, pad at break*=1mm,colback=cellbackground, colframe=cellborder]
\prompt{In}{incolor}{14}{\boxspacing}
\begin{Verbatim}[commandchars=\\\{\}]
\PY{n}{tf}\PY{o}{.}\PY{n}{random}\PY{o}{.}\PY{n}{set\PYZus{}seed}\PY{p}{(}\PY{l+m+mi}{1}\PY{p}{)}
\PY{n}{model}\PY{o}{.}\PY{n}{fit}\PY{p}{(}\PY{p}{[}\PY{n}{user\PYZus{}train}\PY{p}{[}\PY{p}{:}\PY{p}{,} \PY{n}{u\PYZus{}s}\PY{p}{:}\PY{p}{]}\PY{p}{,} \PY{n}{item\PYZus{}train}\PY{p}{[}\PY{p}{:}\PY{p}{,} \PY{n}{i\PYZus{}s}\PY{p}{:}\PY{p}{]}\PY{p}{]}\PY{p}{,} \PY{n}{y\PYZus{}train}\PY{p}{,} \PY{n}{epochs}\PY{o}{=}\PY{l+m+mi}{30}\PY{p}{)}
\end{Verbatim}
\end{tcolorbox}

    \begin{Verbatim}[commandchars=\\\{\}]
Train on 40707 samples
Epoch 1/30
40707/40707 [==============================] - 5s 123us/sample - loss: 0.1232
Epoch 2/30
40707/40707 [==============================] - 5s 111us/sample - loss: 0.1146
Epoch 3/30
40707/40707 [==============================] - 5s 113us/sample - loss: 0.1089
Epoch 4/30
40707/40707 [==============================] - 5s 112us/sample - loss: 0.1039
Epoch 5/30
40707/40707 [==============================] - 5s 113us/sample - loss: 0.1001
Epoch 6/30
40707/40707 [==============================] - 5s 111us/sample - loss: 0.0973
Epoch 7/30
40707/40707 [==============================] - 5s 113us/sample - loss: 0.0956
Epoch 8/30
40707/40707 [==============================] - 5s 112us/sample - loss: 0.0935
Epoch 9/30
40707/40707 [==============================] - 5s 111us/sample - loss: 0.0916
Epoch 10/30
40707/40707 [==============================] - 5s 113us/sample - loss: 0.0897
Epoch 11/30
40707/40707 [==============================] - 5s 112us/sample - loss: 0.0880
Epoch 12/30
40707/40707 [==============================] - 5s 111us/sample - loss: 0.0865
Epoch 13/30
40707/40707 [==============================] - 5s 113us/sample - loss: 0.0852
Epoch 14/30
40707/40707 [==============================] - 5s 111us/sample - loss: 0.0839
Epoch 15/30
40707/40707 [==============================] - 5s 113us/sample - loss: 0.0830
Epoch 16/30
40707/40707 [==============================] - 5s 115us/sample - loss: 0.0815
Epoch 17/30
40707/40707 [==============================] - 5s 113us/sample - loss: 0.0807
Epoch 18/30
40707/40707 [==============================] - 5s 111us/sample - loss: 0.0796
Epoch 19/30
40707/40707 [==============================] - 5s 112us/sample - loss: 0.0786
Epoch 20/30
40707/40707 [==============================] - 5s 113us/sample - loss: 0.0776
Epoch 21/30
40707/40707 [==============================] - 5s 114us/sample - loss: 0.0769
Epoch 22/30
40707/40707 [==============================] - 5s 113us/sample - loss: 0.0761
Epoch 23/30
40707/40707 [==============================] - 5s 112us/sample - loss: 0.0755
Epoch 24/30
40707/40707 [==============================] - 5s 111us/sample - loss: 0.0746
Epoch 25/30
40707/40707 [==============================] - 5s 113us/sample - loss: 0.0741
Epoch 26/30
40707/40707 [==============================] - 5s 112us/sample - loss: 0.0733
Epoch 27/30
40707/40707 [==============================] - 5s 111us/sample - loss: 0.0728
Epoch 28/30
40707/40707 [==============================] - 5s 111us/sample - loss: 0.0723
Epoch 29/30
40707/40707 [==============================] - 5s 111us/sample - loss: 0.0717
Epoch 30/30
40707/40707 [==============================] - 5s 112us/sample - loss: 0.0713
    \end{Verbatim}

            \begin{tcolorbox}[breakable, size=fbox, boxrule=.5pt, pad at break*=1mm, opacityfill=0]
\prompt{Out}{outcolor}{14}{\boxspacing}
\begin{Verbatim}[commandchars=\\\{\}]
<tensorflow.python.keras.callbacks.History at 0x77baffc3d550>
\end{Verbatim}
\end{tcolorbox}
        
    Evaluate the model to determine loss on the test data.

    \begin{tcolorbox}[breakable, size=fbox, boxrule=1pt, pad at break*=1mm,colback=cellbackground, colframe=cellborder]
\prompt{In}{incolor}{15}{\boxspacing}
\begin{Verbatim}[commandchars=\\\{\}]
\PY{n}{model}\PY{o}{.}\PY{n}{evaluate}\PY{p}{(}\PY{p}{[}\PY{n}{user\PYZus{}test}\PY{p}{[}\PY{p}{:}\PY{p}{,} \PY{n}{u\PYZus{}s}\PY{p}{:}\PY{p}{]}\PY{p}{,} \PY{n}{item\PYZus{}test}\PY{p}{[}\PY{p}{:}\PY{p}{,} \PY{n}{i\PYZus{}s}\PY{p}{:}\PY{p}{]}\PY{p}{]}\PY{p}{,} \PY{n}{y\PYZus{}test}\PY{p}{)}
\end{Verbatim}
\end{tcolorbox}

    \begin{Verbatim}[commandchars=\\\{\}]
10177/10177 [==============================] - 0s 35us/sample - loss: 0.0815
    \end{Verbatim}

            \begin{tcolorbox}[breakable, size=fbox, boxrule=.5pt, pad at break*=1mm, opacityfill=0]
\prompt{Out}{outcolor}{15}{\boxspacing}
\begin{Verbatim}[commandchars=\\\{\}]
0.08146006993124337
\end{Verbatim}
\end{tcolorbox}
        
    It is comparable to the training loss indicating the model has not
substantially overfit the training data.

    \#\# 5 - Predictions Below, you'll use your model to make predictions in
a number of circumstances. \#\#\# 5.1 - Predictions for a new user
First, we'll create a new user and have the model suggest movies for
that user. After you have tried this on the example user content, feel
free to change the user content to match your own preferences and see
what the model suggests. Note that ratings are between 0.5 and 5.0,
inclusive, in half-step increments.

    \begin{tcolorbox}[breakable, size=fbox, boxrule=1pt, pad at break*=1mm,colback=cellbackground, colframe=cellborder]
\prompt{In}{incolor}{16}{\boxspacing}
\begin{Verbatim}[commandchars=\\\{\}]
\PY{n}{new\PYZus{}user\PYZus{}id} \PY{o}{=} \PY{l+m+mi}{5000}
\PY{n}{new\PYZus{}rating\PYZus{}ave} \PY{o}{=} \PY{l+m+mf}{0.0}
\PY{n}{new\PYZus{}action} \PY{o}{=} \PY{l+m+mf}{0.0}
\PY{n}{new\PYZus{}adventure} \PY{o}{=} \PY{l+m+mf}{5.0}
\PY{n}{new\PYZus{}animation} \PY{o}{=} \PY{l+m+mf}{0.0}
\PY{n}{new\PYZus{}childrens} \PY{o}{=} \PY{l+m+mf}{0.0}
\PY{n}{new\PYZus{}comedy} \PY{o}{=} \PY{l+m+mf}{0.0}
\PY{n}{new\PYZus{}crime} \PY{o}{=} \PY{l+m+mf}{0.0}
\PY{n}{new\PYZus{}documentary} \PY{o}{=} \PY{l+m+mf}{0.0}
\PY{n}{new\PYZus{}drama} \PY{o}{=} \PY{l+m+mf}{0.0}
\PY{n}{new\PYZus{}fantasy} \PY{o}{=} \PY{l+m+mf}{5.0}
\PY{n}{new\PYZus{}horror} \PY{o}{=} \PY{l+m+mf}{0.0}
\PY{n}{new\PYZus{}mystery} \PY{o}{=} \PY{l+m+mf}{0.0}
\PY{n}{new\PYZus{}romance} \PY{o}{=} \PY{l+m+mf}{0.0}
\PY{n}{new\PYZus{}scifi} \PY{o}{=} \PY{l+m+mf}{0.0}
\PY{n}{new\PYZus{}thriller} \PY{o}{=} \PY{l+m+mf}{0.0}
\PY{n}{new\PYZus{}rating\PYZus{}count} \PY{o}{=} \PY{l+m+mi}{3}

\PY{n}{user\PYZus{}vec} \PY{o}{=} \PY{n}{np}\PY{o}{.}\PY{n}{array}\PY{p}{(}\PY{p}{[}\PY{p}{[}\PY{n}{new\PYZus{}user\PYZus{}id}\PY{p}{,} \PY{n}{new\PYZus{}rating\PYZus{}count}\PY{p}{,} \PY{n}{new\PYZus{}rating\PYZus{}ave}\PY{p}{,}
                      \PY{n}{new\PYZus{}action}\PY{p}{,} \PY{n}{new\PYZus{}adventure}\PY{p}{,} \PY{n}{new\PYZus{}animation}\PY{p}{,} \PY{n}{new\PYZus{}childrens}\PY{p}{,}
                      \PY{n}{new\PYZus{}comedy}\PY{p}{,} \PY{n}{new\PYZus{}crime}\PY{p}{,} \PY{n}{new\PYZus{}documentary}\PY{p}{,}
                      \PY{n}{new\PYZus{}drama}\PY{p}{,} \PY{n}{new\PYZus{}fantasy}\PY{p}{,} \PY{n}{new\PYZus{}horror}\PY{p}{,} \PY{n}{new\PYZus{}mystery}\PY{p}{,}
                      \PY{n}{new\PYZus{}romance}\PY{p}{,} \PY{n}{new\PYZus{}scifi}\PY{p}{,} \PY{n}{new\PYZus{}thriller}\PY{p}{]}\PY{p}{]}\PY{p}{)}
\end{Verbatim}
\end{tcolorbox}

    The new user enjoys movies from the adventure, fantasy genres. Let's
find the top-rated movies for the new user.\\
Below, we'll use a set of movie/item vectors, \texttt{item\_vecs} that
have a vector for each movie in the training/test set. This is matched
with the new user vector above and the scaled vectors are used to
predict ratings for all the movies.

    \begin{tcolorbox}[breakable, size=fbox, boxrule=1pt, pad at break*=1mm,colback=cellbackground, colframe=cellborder]
\prompt{In}{incolor}{17}{\boxspacing}
\begin{Verbatim}[commandchars=\\\{\}]
\PY{c+c1}{\PYZsh{} generate and replicate the user vector to match the number movies in the data set.}
\PY{n}{user\PYZus{}vecs} \PY{o}{=} \PY{n}{gen\PYZus{}user\PYZus{}vecs}\PY{p}{(}\PY{n}{user\PYZus{}vec}\PY{p}{,}\PY{n+nb}{len}\PY{p}{(}\PY{n}{item\PYZus{}vecs}\PY{p}{)}\PY{p}{)}

\PY{c+c1}{\PYZsh{} scale our user and item vectors}
\PY{n}{suser\PYZus{}vecs} \PY{o}{=} \PY{n}{scalerUser}\PY{o}{.}\PY{n}{transform}\PY{p}{(}\PY{n}{user\PYZus{}vecs}\PY{p}{)}
\PY{n}{sitem\PYZus{}vecs} \PY{o}{=} \PY{n}{scalerItem}\PY{o}{.}\PY{n}{transform}\PY{p}{(}\PY{n}{item\PYZus{}vecs}\PY{p}{)}

\PY{c+c1}{\PYZsh{} make a prediction}
\PY{n}{y\PYZus{}p} \PY{o}{=} \PY{n}{model}\PY{o}{.}\PY{n}{predict}\PY{p}{(}\PY{p}{[}\PY{n}{suser\PYZus{}vecs}\PY{p}{[}\PY{p}{:}\PY{p}{,} \PY{n}{u\PYZus{}s}\PY{p}{:}\PY{p}{]}\PY{p}{,} \PY{n}{sitem\PYZus{}vecs}\PY{p}{[}\PY{p}{:}\PY{p}{,} \PY{n}{i\PYZus{}s}\PY{p}{:}\PY{p}{]}\PY{p}{]}\PY{p}{)}

\PY{c+c1}{\PYZsh{} unscale y prediction }
\PY{n}{y\PYZus{}pu} \PY{o}{=} \PY{n}{scalerTarget}\PY{o}{.}\PY{n}{inverse\PYZus{}transform}\PY{p}{(}\PY{n}{y\PYZus{}p}\PY{p}{)}

\PY{c+c1}{\PYZsh{} sort the results, highest prediction first}
\PY{n}{sorted\PYZus{}index} \PY{o}{=} \PY{n}{np}\PY{o}{.}\PY{n}{argsort}\PY{p}{(}\PY{o}{\PYZhy{}}\PY{n}{y\PYZus{}pu}\PY{p}{,}\PY{n}{axis}\PY{o}{=}\PY{l+m+mi}{0}\PY{p}{)}\PY{o}{.}\PY{n}{reshape}\PY{p}{(}\PY{o}{\PYZhy{}}\PY{l+m+mi}{1}\PY{p}{)}\PY{o}{.}\PY{n}{tolist}\PY{p}{(}\PY{p}{)}  \PY{c+c1}{\PYZsh{}negate to get largest rating first}
\PY{n}{sorted\PYZus{}ypu}   \PY{o}{=} \PY{n}{y\PYZus{}pu}\PY{p}{[}\PY{n}{sorted\PYZus{}index}\PY{p}{]}
\PY{n}{sorted\PYZus{}items} \PY{o}{=} \PY{n}{item\PYZus{}vecs}\PY{p}{[}\PY{n}{sorted\PYZus{}index}\PY{p}{]}  \PY{c+c1}{\PYZsh{}using unscaled vectors for display}

\PY{n}{print\PYZus{}pred\PYZus{}movies}\PY{p}{(}\PY{n}{sorted\PYZus{}ypu}\PY{p}{,} \PY{n}{sorted\PYZus{}items}\PY{p}{,} \PY{n}{movie\PYZus{}dict}\PY{p}{,} \PY{n}{maxcount} \PY{o}{=} \PY{l+m+mi}{10}\PY{p}{)}
\end{Verbatim}
\end{tcolorbox}

            \begin{tcolorbox}[breakable, size=fbox, boxrule=.5pt, pad at break*=1mm, opacityfill=0]
\prompt{Out}{outcolor}{17}{\boxspacing}
\begin{Verbatim}[commandchars=\\\{\}]
'<table>\textbackslash{}n<thead>\textbackslash{}n<tr><th style="text-align: right;">  y\_p</th><th style="text-
align: right;">  movie id</th><th style="text-align: right;">  rating
ave</th><th>title                                              </th><th>genres
</th></tr>\textbackslash{}n</thead>\textbackslash{}n<tbody>\textbackslash{}n<tr><td style="text-align: right;">  4.5</td><td
style="text-align: right;">     98809</td><td style="text-align: right;">
3.8</td><td>Hobbit: An Unexpected Journey, The (2012)
</td><td>Adventure|Fantasy               </td></tr>\textbackslash{}n<tr><td style="text-align:
right;">  4.4</td><td style="text-align: right;">      8368</td><td style="text-
align: right;">         3.9</td><td>Harry Potter and the Prisoner of Azkaban
(2004)    </td><td>Adventure|Fantasy               </td></tr>\textbackslash{}n<tr><td
style="text-align: right;">  4.4</td><td style="text-align: right;">
54001</td><td style="text-align: right;">         3.9</td><td>Harry Potter and
the Order of the Phoenix (2007)   </td><td>Adventure|Drama|Fantasy
</td></tr>\textbackslash{}n<tr><td style="text-align: right;">  4.3</td><td style="text-align:
right;">     40815</td><td style="text-align: right;">         3.8</td><td>Harry
Potter and the Goblet of Fire (2005)         </td><td>Adventure|Fantasy|Thriller
</td></tr>\textbackslash{}n<tr><td style="text-align: right;">  4.3</td><td style="text-align:
right;">    106489</td><td style="text-align: right;">
3.6</td><td>Hobbit: The Desolation of Smaug, The (2013)
</td><td>Adventure|Fantasy               </td></tr>\textbackslash{}n<tr><td style="text-align:
right;">  4.3</td><td style="text-align: right;">     81834</td><td style="text-
align: right;">         4  </td><td>Harry Potter and the Deathly Hallows: Part 1
(2010)</td><td>Action|Adventure|Fantasy        </td></tr>\textbackslash{}n<tr><td style="text-
align: right;">  4.3</td><td style="text-align: right;">     59387</td><td
style="text-align: right;">         4  </td><td>Fall, The (2006)
</td><td>Adventure|Drama|Fantasy         </td></tr>\textbackslash{}n<tr><td style="text-align:
right;">  4.3</td><td style="text-align: right;">      5952</td><td style="text-
align: right;">         4  </td><td>Lord of the Rings: The Two Towers, The
(2002)      </td><td>Adventure|Fantasy               </td></tr>\textbackslash{}n<tr><td
style="text-align: right;">  4.3</td><td style="text-align: right;">
5816</td><td style="text-align: right;">         3.6</td><td>Harry Potter and
the Chamber of Secrets (2002)     </td><td>Adventure|Fantasy
</td></tr>\textbackslash{}n<tr><td style="text-align: right;">  4.3</td><td style="text-align:
right;">     54259</td><td style="text-align: right;">
3.6</td><td>Stardust (2007)
</td><td>Adventure|Comedy|Fantasy|Romance</td></tr>\textbackslash{}n</tbody>\textbackslash{}n</table>'
\end{Verbatim}
\end{tcolorbox}
        
    \#\#\# 5.2 - Predictions for an existing user. Let's look at the
predictions for ``user 2'', one of the users in the data set. We can
compare the predicted ratings with the model's ratings.

    \begin{tcolorbox}[breakable, size=fbox, boxrule=1pt, pad at break*=1mm,colback=cellbackground, colframe=cellborder]
\prompt{In}{incolor}{18}{\boxspacing}
\begin{Verbatim}[commandchars=\\\{\}]
\PY{n}{uid} \PY{o}{=} \PY{l+m+mi}{2} 
\PY{c+c1}{\PYZsh{} form a set of user vectors. This is the same vector, transformed and repeated.}
\PY{n}{user\PYZus{}vecs}\PY{p}{,} \PY{n}{y\PYZus{}vecs} \PY{o}{=} \PY{n}{get\PYZus{}user\PYZus{}vecs}\PY{p}{(}\PY{n}{uid}\PY{p}{,} \PY{n}{user\PYZus{}train\PYZus{}unscaled}\PY{p}{,} \PY{n}{item\PYZus{}vecs}\PY{p}{,} \PY{n}{user\PYZus{}to\PYZus{}genre}\PY{p}{)}

\PY{c+c1}{\PYZsh{} scale our user and item vectors}
\PY{n}{suser\PYZus{}vecs} \PY{o}{=} \PY{n}{scalerUser}\PY{o}{.}\PY{n}{transform}\PY{p}{(}\PY{n}{user\PYZus{}vecs}\PY{p}{)}
\PY{n}{sitem\PYZus{}vecs} \PY{o}{=} \PY{n}{scalerItem}\PY{o}{.}\PY{n}{transform}\PY{p}{(}\PY{n}{item\PYZus{}vecs}\PY{p}{)}

\PY{c+c1}{\PYZsh{} make a prediction}
\PY{n}{y\PYZus{}p} \PY{o}{=} \PY{n}{model}\PY{o}{.}\PY{n}{predict}\PY{p}{(}\PY{p}{[}\PY{n}{suser\PYZus{}vecs}\PY{p}{[}\PY{p}{:}\PY{p}{,} \PY{n}{u\PYZus{}s}\PY{p}{:}\PY{p}{]}\PY{p}{,} \PY{n}{sitem\PYZus{}vecs}\PY{p}{[}\PY{p}{:}\PY{p}{,} \PY{n}{i\PYZus{}s}\PY{p}{:}\PY{p}{]}\PY{p}{]}\PY{p}{)}

\PY{c+c1}{\PYZsh{} unscale y prediction }
\PY{n}{y\PYZus{}pu} \PY{o}{=} \PY{n}{scalerTarget}\PY{o}{.}\PY{n}{inverse\PYZus{}transform}\PY{p}{(}\PY{n}{y\PYZus{}p}\PY{p}{)}

\PY{c+c1}{\PYZsh{} sort the results, highest prediction first}
\PY{n}{sorted\PYZus{}index} \PY{o}{=} \PY{n}{np}\PY{o}{.}\PY{n}{argsort}\PY{p}{(}\PY{o}{\PYZhy{}}\PY{n}{y\PYZus{}pu}\PY{p}{,}\PY{n}{axis}\PY{o}{=}\PY{l+m+mi}{0}\PY{p}{)}\PY{o}{.}\PY{n}{reshape}\PY{p}{(}\PY{o}{\PYZhy{}}\PY{l+m+mi}{1}\PY{p}{)}\PY{o}{.}\PY{n}{tolist}\PY{p}{(}\PY{p}{)}  \PY{c+c1}{\PYZsh{}negate to get largest rating first}
\PY{n}{sorted\PYZus{}ypu}   \PY{o}{=} \PY{n}{y\PYZus{}pu}\PY{p}{[}\PY{n}{sorted\PYZus{}index}\PY{p}{]}
\PY{n}{sorted\PYZus{}items} \PY{o}{=} \PY{n}{item\PYZus{}vecs}\PY{p}{[}\PY{n}{sorted\PYZus{}index}\PY{p}{]}  \PY{c+c1}{\PYZsh{}using unscaled vectors for display}
\PY{n}{sorted\PYZus{}user}  \PY{o}{=} \PY{n}{user\PYZus{}vecs}\PY{p}{[}\PY{n}{sorted\PYZus{}index}\PY{p}{]}
\PY{n}{sorted\PYZus{}y}     \PY{o}{=} \PY{n}{y\PYZus{}vecs}\PY{p}{[}\PY{n}{sorted\PYZus{}index}\PY{p}{]}

\PY{c+c1}{\PYZsh{}print sorted predictions for movies rated by the user}
\PY{n}{print\PYZus{}existing\PYZus{}user}\PY{p}{(}\PY{n}{sorted\PYZus{}ypu}\PY{p}{,} \PY{n}{sorted\PYZus{}y}\PY{o}{.}\PY{n}{reshape}\PY{p}{(}\PY{o}{\PYZhy{}}\PY{l+m+mi}{1}\PY{p}{,}\PY{l+m+mi}{1}\PY{p}{)}\PY{p}{,} \PY{n}{sorted\PYZus{}user}\PY{p}{,} \PY{n}{sorted\PYZus{}items}\PY{p}{,} \PY{n}{ivs}\PY{p}{,} \PY{n}{uvs}\PY{p}{,} \PY{n}{movie\PYZus{}dict}\PY{p}{,} \PY{n}{maxcount} \PY{o}{=} \PY{l+m+mi}{50}\PY{p}{)}
\end{Verbatim}
\end{tcolorbox}

            \begin{tcolorbox}[breakable, size=fbox, boxrule=.5pt, pad at break*=1mm, opacityfill=0]
\prompt{Out}{outcolor}{18}{\boxspacing}
\begin{Verbatim}[commandchars=\\\{\}]
'<table>\textbackslash{}n<thead>\textbackslash{}n<tr><th style="text-align: right;">  y\_p</th><th style="text-
align: right;">  y</th><th style="text-align: right;">  user</th><th>user genre
ave           </th><th style="text-align: right;">  movie rating ave</th><th
style="text-align: right;">  movie id</th><th>title
</th><th>genres
</th></tr>\textbackslash{}n</thead>\textbackslash{}n<tbody>\textbackslash{}n<tr><td style="text-align: right;">  4.5</td><td
style="text-align: right;">5.0</td><td style="text-align: right;">
2</td><td>[4.0]                    </td><td style="text-align: right;">
4.3</td><td style="text-align: right;">     80906</td><td>Inside Job (2010)
</td><td>Documentary                               </td></tr>\textbackslash{}n<tr><td
style="text-align: right;">  4.2</td><td style="text-align: right;">3.5</td><td
style="text-align: right;">     2</td><td>[4.0,4.0]                </td><td
style="text-align: right;">               3.9</td><td style="text-align:
right;">     99114</td><td>Django Unchained (2012)
</td><td>Action|Drama                              </td></tr>\textbackslash{}n<tr><td
style="text-align: right;">  4.1</td><td style="text-align: right;">4.5</td><td
style="text-align: right;">     2</td><td>[4.0,4.0]                </td><td
style="text-align: right;">               4.1</td><td style="text-align:
right;">     68157</td><td>Inglourious Basterds (2009)
</td><td>Action|Drama                              </td></tr>\textbackslash{}n<tr><td
style="text-align: right;">  4.1</td><td style="text-align: right;">3.5</td><td
style="text-align: right;">     2</td><td>[4.0,3.9,3.9]            </td><td
style="text-align: right;">               3.9</td><td style="text-align:
right;">    115713</td><td>Ex Machina (2015)
</td><td>Drama|Sci-Fi|Thriller                     </td></tr>\textbackslash{}n<tr><td
style="text-align: right;">  4.0</td><td style="text-align: right;">4.0</td><td
style="text-align: right;">     2</td><td>[4.0,4.1,4.0,4.0,3.9,3.9]</td><td
style="text-align: right;">               4.1</td><td style="text-align:
right;">     79132</td><td>Inception (2010)
</td><td>Action|Crime|Drama|Mystery|Sci-Fi|Thriller</td></tr>\textbackslash{}n<tr><td
style="text-align: right;">  4.0</td><td style="text-align: right;">4.0</td><td
style="text-align: right;">     2</td><td>[4.1,4.0,3.9]            </td><td
style="text-align: right;">               4.3</td><td style="text-align:
right;">     48516</td><td>Departed, The (2006)
</td><td>Crime|Drama|Thriller                      </td></tr>\textbackslash{}n<tr><td
style="text-align: right;">  4.0</td><td style="text-align: right;">4.5</td><td
style="text-align: right;">     2</td><td>[4.0,4.1,4.0]            </td><td
style="text-align: right;">               4.2</td><td style="text-align:
right;">     58559</td><td>Dark Knight, The (2008)
</td><td>Action|Crime|Drama                        </td></tr>\textbackslash{}n<tr><td
style="text-align: right;">  4.0</td><td style="text-align: right;">4.0</td><td
style="text-align: right;">     2</td><td>[4.0,4.1,3.9]            </td><td
style="text-align: right;">               4.0</td><td style="text-align:
right;">      6874</td><td>Kill Bill: Vol. 1 (2003)
</td><td>Action|Crime|Thriller                     </td></tr>\textbackslash{}n<tr><td
style="text-align: right;">  4.0</td><td style="text-align: right;">3.5</td><td
style="text-align: right;">     2</td><td>[4.0,4.1,4.0,3.9]        </td><td
style="text-align: right;">               3.8</td><td style="text-align:
right;">      8798</td><td>Collateral (2004)
</td><td>Action|Crime|Drama|Thriller               </td></tr>\textbackslash{}n<tr><td
style="text-align: right;">  3.9</td><td style="text-align: right;">5.0</td><td
style="text-align: right;">     2</td><td>[4.0,4.1,4.0]            </td><td
style="text-align: right;">               3.9</td><td style="text-align:
right;">    106782</td><td>Wolf of Wall Street, The (2013)
</td><td>Comedy|Crime|Drama                        </td></tr>\textbackslash{}n<tr><td
style="text-align: right;">  3.9</td><td style="text-align: right;">3.5</td><td
style="text-align: right;">     2</td><td>[4.0,4.2,4.1]            </td><td
style="text-align: right;">               4.0</td><td style="text-align:
right;">     91529</td><td>Dark Knight Rises, The (2012)
</td><td>Action|Adventure|Crime                    </td></tr>\textbackslash{}n<tr><td
style="text-align: right;">  3.9</td><td style="text-align: right;">4.0</td><td
style="text-align: right;">     2</td><td>[4.0,4.0,3.9]            </td><td
style="text-align: right;">               4.0</td><td style="text-align:
right;">     74458</td><td>Shutter Island (2010)
</td><td>Drama|Mystery|Thriller                    </td></tr>\textbackslash{}n<tr><td
style="text-align: right;">  3.9</td><td style="text-align: right;">4.5</td><td
style="text-align: right;">     2</td><td>[4.1,4.0,3.9]            </td><td
style="text-align: right;">               4.0</td><td style="text-align:
right;">     80489</td><td>Town, The (2010)
</td><td>Crime|Drama|Thriller                      </td></tr>\textbackslash{}n<tr><td
style="text-align: right;">  3.8</td><td style="text-align: right;">4.0</td><td
style="text-align: right;">     2</td><td>[4.0]                    </td><td
style="text-align: right;">               4.0</td><td style="text-align:
right;">    112552</td><td>Whiplash (2014)
</td><td>Drama                                     </td></tr>\textbackslash{}n<tr><td
style="text-align: right;">  3.8</td><td style="text-align: right;">3.0</td><td
style="text-align: right;">     2</td><td>[3.9]                    </td><td
style="text-align: right;">               4.0</td><td style="text-align:
right;">    109487</td><td>Interstellar (2014)
</td><td>Sci-Fi                                    </td></tr>\textbackslash{}n<tr><td
style="text-align: right;">  3.8</td><td style="text-align: right;">5.0</td><td
style="text-align: right;">     2</td><td>[4.0]                    </td><td
style="text-align: right;">               3.7</td><td style="text-align:
right;">     89774</td><td>Warrior (2011)
</td><td>Drama                                     </td></tr>\textbackslash{}n<tr><td
style="text-align: right;">  3.7</td><td style="text-align: right;">3.0</td><td
style="text-align: right;">     2</td><td>[4.0,4.0,3.0]            </td><td
style="text-align: right;">               3.9</td><td style="text-align:
right;">     71535</td><td>Zombieland (2009)
</td><td>Action|Comedy|Horror                      </td></tr>\textbackslash{}n<tr><td
style="text-align: right;">  3.7</td><td style="text-align: right;">5.0</td><td
style="text-align: right;">     2</td><td>[4.0,4.2,3.9,3.9]        </td><td
style="text-align: right;">               3.8</td><td style="text-align:
right;">    122882</td><td>Mad Max: Fury Road (2015)
</td><td>Action|Adventure|Sci-Fi|Thriller          </td></tr>\textbackslash{}n<tr><td
style="text-align: right;">  3.5</td><td style="text-align: right;">5.0</td><td
style="text-align: right;">     2</td><td>[4.0]                    </td><td
style="text-align: right;">               3.6</td><td style="text-align:
right;">     60756</td><td>Step Brothers (2008)
</td><td>Comedy                                    </td></tr>\textbackslash{}n<tr><td
style="text-align: right;">  3.5</td><td style="text-align: right;">2.5</td><td
style="text-align: right;">     2</td><td>[4.0,3.9]                </td><td
style="text-align: right;">               3.5</td><td style="text-align:
right;">     91658</td><td>Girl with the Dragon Tattoo, The (2011)
</td><td>Drama|Thriller                            </td></tr>\textbackslash{}n<tr><td
style="text-align: right;">  3.1</td><td style="text-align: right;">3.0</td><td
style="text-align: right;">     2</td><td>[4.0,4.0]                </td><td
style="text-align: right;">               4.0</td><td style="text-align:
right;">     77455</td><td>Exit Through the Gift Shop (2010)
</td><td>Comedy|Documentary                        </td></tr>\textbackslash{}n<tr><td
style="text-align: right;">  3.1</td><td style="text-align: right;">4.0</td><td
style="text-align: right;">     2</td><td>[4.0,4.0]                </td><td
style="text-align: right;">               3.2</td><td style="text-align:
right;">     46970</td><td>Talladega Nights: The Ballad of Ricky Bobby
(2006)</td><td>Action|Comedy
</td></tr>\textbackslash{}n</tbody>\textbackslash{}n</table>'
\end{Verbatim}
\end{tcolorbox}
        
    The model prediction is generally within 1 of the actual rating though
it is not a very accurate predictor of how a user rates specific movies.
This is especially true if the user rating is significantly different
than the user's genre average. You can vary the user id above to try
different users. Not all user id's were used in the training set.

    \#\#\# 5.3 - Finding Similar Items The neural network above produces two
feature vectors, a user feature vector \(v_u\), and a movie feature
vector, \(v_m\). These are 32 entry vectors whose values are difficult
to interpret. However, similar items will have similar vectors. This
information can be used to make recommendations. For example, if a user
has rated ``Toy Story 3'' highly, one could recommend similar movies by
selecting movies with similar movie feature vectors.

A similarity measure is the squared distance between the two vectors \$
\mathbf{v_m^{(k)}}\$ and \(\mathbf{v_m^{(i)}}\) :
\[\left\Vert \mathbf{v_m^{(k)}} - \mathbf{v_m^{(i)}}  \right\Vert^2 = \sum_{l=1}^{n}(v_{m_l}^{(k)} - v_{m_l}^{(i)})^2\tag{1}\]

    \#\#\# Exercise 2

Write a function to compute the square distance.

    \begin{tcolorbox}[breakable, size=fbox, boxrule=1pt, pad at break*=1mm,colback=cellbackground, colframe=cellborder]
\prompt{In}{incolor}{19}{\boxspacing}
\begin{Verbatim}[commandchars=\\\{\}]
\PY{c+c1}{\PYZsh{} GRADED\PYZus{}FUNCTION: sq\PYZus{}dist}
\PY{c+c1}{\PYZsh{} UNQ\PYZus{}C2}
\PY{k}{def} \PY{n+nf}{sq\PYZus{}dist}\PY{p}{(}\PY{n}{a}\PY{p}{,}\PY{n}{b}\PY{p}{)}\PY{p}{:}
    \PY{l+s+sd}{\PYZdq{}\PYZdq{}\PYZdq{}}
\PY{l+s+sd}{    Returns the squared distance between two vectors}
\PY{l+s+sd}{    Args:}
\PY{l+s+sd}{      a (ndarray (n,)): vector with n features}
\PY{l+s+sd}{      b (ndarray (n,)): vector with n features}
\PY{l+s+sd}{    Returns:}
\PY{l+s+sd}{      d (float) : distance}
\PY{l+s+sd}{    \PYZdq{}\PYZdq{}\PYZdq{}}
    \PY{c+c1}{\PYZsh{}\PYZsh{}\PYZsh{} START CODE HERE \PYZsh{}\PYZsh{}\PYZsh{}     }
    \PY{n}{diff} \PY{o}{=} \PY{n}{a} \PY{o}{\PYZhy{}} \PY{n}{b}
    \PY{n}{d} \PY{o}{=} \PY{n}{np}\PY{o}{.}\PY{n}{sum}\PY{p}{(}\PY{n}{diff}\PY{o}{*}\PY{o}{*}\PY{l+m+mi}{2}\PY{p}{)}
    \PY{c+c1}{\PYZsh{}\PYZsh{}\PYZsh{} END CODE HERE \PYZsh{}\PYZsh{}\PYZsh{}     }
    \PY{k}{return} \PY{n}{d}
\end{Verbatim}
\end{tcolorbox}

    \begin{tcolorbox}[breakable, size=fbox, boxrule=1pt, pad at break*=1mm,colback=cellbackground, colframe=cellborder]
\prompt{In}{incolor}{20}{\boxspacing}
\begin{Verbatim}[commandchars=\\\{\}]
\PY{n}{a1} \PY{o}{=} \PY{n}{np}\PY{o}{.}\PY{n}{array}\PY{p}{(}\PY{p}{[}\PY{l+m+mf}{1.0}\PY{p}{,} \PY{l+m+mf}{2.0}\PY{p}{,} \PY{l+m+mf}{3.0}\PY{p}{]}\PY{p}{)}\PY{p}{;} \PY{n}{b1} \PY{o}{=} \PY{n}{np}\PY{o}{.}\PY{n}{array}\PY{p}{(}\PY{p}{[}\PY{l+m+mf}{1.0}\PY{p}{,} \PY{l+m+mf}{2.0}\PY{p}{,} \PY{l+m+mf}{3.0}\PY{p}{]}\PY{p}{)}
\PY{n}{a2} \PY{o}{=} \PY{n}{np}\PY{o}{.}\PY{n}{array}\PY{p}{(}\PY{p}{[}\PY{l+m+mf}{1.1}\PY{p}{,} \PY{l+m+mf}{2.1}\PY{p}{,} \PY{l+m+mf}{3.1}\PY{p}{]}\PY{p}{)}\PY{p}{;} \PY{n}{b2} \PY{o}{=} \PY{n}{np}\PY{o}{.}\PY{n}{array}\PY{p}{(}\PY{p}{[}\PY{l+m+mf}{1.0}\PY{p}{,} \PY{l+m+mf}{2.0}\PY{p}{,} \PY{l+m+mf}{3.0}\PY{p}{]}\PY{p}{)}
\PY{n}{a3} \PY{o}{=} \PY{n}{np}\PY{o}{.}\PY{n}{array}\PY{p}{(}\PY{p}{[}\PY{l+m+mi}{0}\PY{p}{,} \PY{l+m+mi}{1}\PY{p}{,} \PY{l+m+mi}{0}\PY{p}{]}\PY{p}{)}\PY{p}{;}       \PY{n}{b3} \PY{o}{=} \PY{n}{np}\PY{o}{.}\PY{n}{array}\PY{p}{(}\PY{p}{[}\PY{l+m+mi}{1}\PY{p}{,} \PY{l+m+mi}{0}\PY{p}{,} \PY{l+m+mi}{0}\PY{p}{]}\PY{p}{)}
\PY{n+nb}{print}\PY{p}{(}\PY{l+s+sa}{f}\PY{l+s+s2}{\PYZdq{}}\PY{l+s+s2}{squared distance between a1 and b1: }\PY{l+s+si}{\PYZob{}}\PY{n}{sq\PYZus{}dist}\PY{p}{(}\PY{n}{a1}\PY{p}{,} \PY{n}{b1}\PY{p}{)}\PY{l+s+si}{:}\PY{l+s+s2}{0.3f}\PY{l+s+si}{\PYZcb{}}\PY{l+s+s2}{\PYZdq{}}\PY{p}{)}
\PY{n+nb}{print}\PY{p}{(}\PY{l+s+sa}{f}\PY{l+s+s2}{\PYZdq{}}\PY{l+s+s2}{squared distance between a2 and b2: }\PY{l+s+si}{\PYZob{}}\PY{n}{sq\PYZus{}dist}\PY{p}{(}\PY{n}{a2}\PY{p}{,} \PY{n}{b2}\PY{p}{)}\PY{l+s+si}{:}\PY{l+s+s2}{0.3f}\PY{l+s+si}{\PYZcb{}}\PY{l+s+s2}{\PYZdq{}}\PY{p}{)}
\PY{n+nb}{print}\PY{p}{(}\PY{l+s+sa}{f}\PY{l+s+s2}{\PYZdq{}}\PY{l+s+s2}{squared distance between a3 and b3: }\PY{l+s+si}{\PYZob{}}\PY{n}{sq\PYZus{}dist}\PY{p}{(}\PY{n}{a3}\PY{p}{,} \PY{n}{b3}\PY{p}{)}\PY{l+s+si}{:}\PY{l+s+s2}{0.3f}\PY{l+s+si}{\PYZcb{}}\PY{l+s+s2}{\PYZdq{}}\PY{p}{)}
\end{Verbatim}
\end{tcolorbox}

    \begin{Verbatim}[commandchars=\\\{\}]
squared distance between a1 and b1: 0.000
squared distance between a2 and b2: 0.030
squared distance between a3 and b3: 2.000
    \end{Verbatim}

    \textbf{Expected Output}:

squared distance between a1 and b1: 0.000\\
squared distance between a2 and b2: 0.030\\
squared distance between a3 and b3: 2.000

    \begin{tcolorbox}[breakable, size=fbox, boxrule=1pt, pad at break*=1mm,colback=cellbackground, colframe=cellborder]
\prompt{In}{incolor}{21}{\boxspacing}
\begin{Verbatim}[commandchars=\\\{\}]
\PY{c+c1}{\PYZsh{} Public tests}
\PY{n}{test\PYZus{}sq\PYZus{}dist}\PY{p}{(}\PY{n}{sq\PYZus{}dist}\PY{p}{)}
\end{Verbatim}
\end{tcolorbox}

    \begin{Verbatim}[commandchars=\\\{\}]
\textcolor{ansi-green-intense}{All tests passed!}
    \end{Verbatim}

    Click for hints

While a summation is often an indication a for loop should be used, here
the subtraction can be element-wise in one statement. Further, you can
utilized np.square to square, element-wise, the result of the
subtraction. np.sum can be used to sum the squared elements.

    A matrix of distances between movies can be computed once when the model
is trained and then reused for new recommendations without retraining.
The first step, once a model is trained, is to obtain the movie feature
vector, \(v_m\), for each of the movies. To do this, we will use the
trained \texttt{item\_NN} and build a small model to allow us to run the
movie vectors through it to generate \(v_m\).

    \begin{tcolorbox}[breakable, size=fbox, boxrule=1pt, pad at break*=1mm,colback=cellbackground, colframe=cellborder]
\prompt{In}{incolor}{22}{\boxspacing}
\begin{Verbatim}[commandchars=\\\{\}]
\PY{n}{input\PYZus{}item\PYZus{}m} \PY{o}{=} \PY{n}{tf}\PY{o}{.}\PY{n}{keras}\PY{o}{.}\PY{n}{layers}\PY{o}{.}\PY{n}{Input}\PY{p}{(}\PY{n}{shape}\PY{o}{=}\PY{p}{(}\PY{n}{num\PYZus{}item\PYZus{}features}\PY{p}{)}\PY{p}{)}    \PY{c+c1}{\PYZsh{} input layer}
\PY{n}{vm\PYZus{}m} \PY{o}{=} \PY{n}{item\PYZus{}NN}\PY{p}{(}\PY{n}{input\PYZus{}item\PYZus{}m}\PY{p}{)}                                       \PY{c+c1}{\PYZsh{} use the trained item\PYZus{}NN}
\PY{n}{vm\PYZus{}m} \PY{o}{=} \PY{n}{tf}\PY{o}{.}\PY{n}{linalg}\PY{o}{.}\PY{n}{l2\PYZus{}normalize}\PY{p}{(}\PY{n}{vm\PYZus{}m}\PY{p}{,} \PY{n}{axis}\PY{o}{=}\PY{l+m+mi}{1}\PY{p}{)}                        \PY{c+c1}{\PYZsh{} incorporate normalization as was done in the original model}
\PY{n}{model\PYZus{}m} \PY{o}{=} \PY{n}{tf}\PY{o}{.}\PY{n}{keras}\PY{o}{.}\PY{n}{Model}\PY{p}{(}\PY{n}{input\PYZus{}item\PYZus{}m}\PY{p}{,} \PY{n}{vm\PYZus{}m}\PY{p}{)}                                
\PY{n}{model\PYZus{}m}\PY{o}{.}\PY{n}{summary}\PY{p}{(}\PY{p}{)}
\end{Verbatim}
\end{tcolorbox}

    \begin{Verbatim}[commandchars=\\\{\}]
Model: "model\_1"
\_\_\_\_\_\_\_\_\_\_\_\_\_\_\_\_\_\_\_\_\_\_\_\_\_\_\_\_\_\_\_\_\_\_\_\_\_\_\_\_\_\_\_\_\_\_\_\_\_\_\_\_\_\_\_\_\_\_\_\_\_\_\_\_\_\_\_\_\_\_\_\_\_\_\_\_\_\_\_\_
\_\_\_\_\_\_\_\_\_\_\_\_\_\_\_\_\_\_
Layer (type)                    Output Shape         Param \#     Connected to
================================================================================
==================
input\_3 (InputLayer)            [(None, 16)]         0
\_\_\_\_\_\_\_\_\_\_\_\_\_\_\_\_\_\_\_\_\_\_\_\_\_\_\_\_\_\_\_\_\_\_\_\_\_\_\_\_\_\_\_\_\_\_\_\_\_\_\_\_\_\_\_\_\_\_\_\_\_\_\_\_\_\_\_\_\_\_\_\_\_\_\_\_\_\_\_\_
\_\_\_\_\_\_\_\_\_\_\_\_\_\_\_\_\_\_
sequential\_1 (Sequential)       (None, 32)           41376       input\_3[0][0]
\_\_\_\_\_\_\_\_\_\_\_\_\_\_\_\_\_\_\_\_\_\_\_\_\_\_\_\_\_\_\_\_\_\_\_\_\_\_\_\_\_\_\_\_\_\_\_\_\_\_\_\_\_\_\_\_\_\_\_\_\_\_\_\_\_\_\_\_\_\_\_\_\_\_\_\_\_\_\_\_
\_\_\_\_\_\_\_\_\_\_\_\_\_\_\_\_\_\_
tf\_op\_layer\_l2\_normalize\_2/Squa [(None, 32)]         0
sequential\_1[1][0]
\_\_\_\_\_\_\_\_\_\_\_\_\_\_\_\_\_\_\_\_\_\_\_\_\_\_\_\_\_\_\_\_\_\_\_\_\_\_\_\_\_\_\_\_\_\_\_\_\_\_\_\_\_\_\_\_\_\_\_\_\_\_\_\_\_\_\_\_\_\_\_\_\_\_\_\_\_\_\_\_
\_\_\_\_\_\_\_\_\_\_\_\_\_\_\_\_\_\_
tf\_op\_layer\_l2\_normalize\_2/Sum  [(None, 1)]          0
tf\_op\_layer\_l2\_normalize\_2/Square
\_\_\_\_\_\_\_\_\_\_\_\_\_\_\_\_\_\_\_\_\_\_\_\_\_\_\_\_\_\_\_\_\_\_\_\_\_\_\_\_\_\_\_\_\_\_\_\_\_\_\_\_\_\_\_\_\_\_\_\_\_\_\_\_\_\_\_\_\_\_\_\_\_\_\_\_\_\_\_\_
\_\_\_\_\_\_\_\_\_\_\_\_\_\_\_\_\_\_
tf\_op\_layer\_l2\_normalize\_2/Maxi [(None, 1)]          0
tf\_op\_layer\_l2\_normalize\_2/Sum[0]
\_\_\_\_\_\_\_\_\_\_\_\_\_\_\_\_\_\_\_\_\_\_\_\_\_\_\_\_\_\_\_\_\_\_\_\_\_\_\_\_\_\_\_\_\_\_\_\_\_\_\_\_\_\_\_\_\_\_\_\_\_\_\_\_\_\_\_\_\_\_\_\_\_\_\_\_\_\_\_\_
\_\_\_\_\_\_\_\_\_\_\_\_\_\_\_\_\_\_
tf\_op\_layer\_l2\_normalize\_2/Rsqr [(None, 1)]          0
tf\_op\_layer\_l2\_normalize\_2/Maximu
\_\_\_\_\_\_\_\_\_\_\_\_\_\_\_\_\_\_\_\_\_\_\_\_\_\_\_\_\_\_\_\_\_\_\_\_\_\_\_\_\_\_\_\_\_\_\_\_\_\_\_\_\_\_\_\_\_\_\_\_\_\_\_\_\_\_\_\_\_\_\_\_\_\_\_\_\_\_\_\_
\_\_\_\_\_\_\_\_\_\_\_\_\_\_\_\_\_\_
tf\_op\_layer\_l2\_normalize\_2 (Ten [(None, 32)]         0
sequential\_1[1][0]
tf\_op\_layer\_l2\_normalize\_2/Rsqrt[
================================================================================
==================
Total params: 41,376
Trainable params: 41,376
Non-trainable params: 0
\_\_\_\_\_\_\_\_\_\_\_\_\_\_\_\_\_\_\_\_\_\_\_\_\_\_\_\_\_\_\_\_\_\_\_\_\_\_\_\_\_\_\_\_\_\_\_\_\_\_\_\_\_\_\_\_\_\_\_\_\_\_\_\_\_\_\_\_\_\_\_\_\_\_\_\_\_\_\_\_
\_\_\_\_\_\_\_\_\_\_\_\_\_\_\_\_\_\_
    \end{Verbatim}

    Once you have a movie model, you can create a set of movie feature
vectors by using the model to predict using a set of item/movie vectors
as input. \texttt{item\_vecs} is a set of all of the movie vectors. It
must be scaled to use with the trained model. The result of the
prediction is a 32 entry feature vector for each movie.

    \begin{tcolorbox}[breakable, size=fbox, boxrule=1pt, pad at break*=1mm,colback=cellbackground, colframe=cellborder]
\prompt{In}{incolor}{23}{\boxspacing}
\begin{Verbatim}[commandchars=\\\{\}]
\PY{n}{scaled\PYZus{}item\PYZus{}vecs} \PY{o}{=} \PY{n}{scalerItem}\PY{o}{.}\PY{n}{transform}\PY{p}{(}\PY{n}{item\PYZus{}vecs}\PY{p}{)}
\PY{n}{vms} \PY{o}{=} \PY{n}{model\PYZus{}m}\PY{o}{.}\PY{n}{predict}\PY{p}{(}\PY{n}{scaled\PYZus{}item\PYZus{}vecs}\PY{p}{[}\PY{p}{:}\PY{p}{,}\PY{n}{i\PYZus{}s}\PY{p}{:}\PY{p}{]}\PY{p}{)}
\PY{n+nb}{print}\PY{p}{(}\PY{l+s+sa}{f}\PY{l+s+s2}{\PYZdq{}}\PY{l+s+s2}{size of all predicted movie feature vectors: }\PY{l+s+si}{\PYZob{}}\PY{n}{vms}\PY{o}{.}\PY{n}{shape}\PY{l+s+si}{\PYZcb{}}\PY{l+s+s2}{\PYZdq{}}\PY{p}{)}
\end{Verbatim}
\end{tcolorbox}

    \begin{Verbatim}[commandchars=\\\{\}]
size of all predicted movie feature vectors: (847, 32)
    \end{Verbatim}

    Let's now compute a matrix of the squared distance between each movie
feature vector and all other movie feature vectors:

    We can then find the closest movie by finding the minimum along each
row. We will make use of
\href{https://numpy.org/doc/1.21/user/tutorial-ma.html}{numpy masked
arrays} to avoid selecting the same movie. The masked values along the
diagonal won't be included in the computation.

    \begin{tcolorbox}[breakable, size=fbox, boxrule=1pt, pad at break*=1mm,colback=cellbackground, colframe=cellborder]
\prompt{In}{incolor}{24}{\boxspacing}
\begin{Verbatim}[commandchars=\\\{\}]
\PY{n}{count} \PY{o}{=} \PY{l+m+mi}{50}  \PY{c+c1}{\PYZsh{} number of movies to display}
\PY{n}{dim} \PY{o}{=} \PY{n+nb}{len}\PY{p}{(}\PY{n}{vms}\PY{p}{)}
\PY{n}{dist} \PY{o}{=} \PY{n}{np}\PY{o}{.}\PY{n}{zeros}\PY{p}{(}\PY{p}{(}\PY{n}{dim}\PY{p}{,}\PY{n}{dim}\PY{p}{)}\PY{p}{)}

\PY{k}{for} \PY{n}{i} \PY{o+ow}{in} \PY{n+nb}{range}\PY{p}{(}\PY{n}{dim}\PY{p}{)}\PY{p}{:}
    \PY{k}{for} \PY{n}{j} \PY{o+ow}{in} \PY{n+nb}{range}\PY{p}{(}\PY{n}{dim}\PY{p}{)}\PY{p}{:}
        \PY{n}{dist}\PY{p}{[}\PY{n}{i}\PY{p}{,}\PY{n}{j}\PY{p}{]} \PY{o}{=} \PY{n}{sq\PYZus{}dist}\PY{p}{(}\PY{n}{vms}\PY{p}{[}\PY{n}{i}\PY{p}{,} \PY{p}{:}\PY{p}{]}\PY{p}{,} \PY{n}{vms}\PY{p}{[}\PY{n}{j}\PY{p}{,} \PY{p}{:}\PY{p}{]}\PY{p}{)}
        
\PY{n}{m\PYZus{}dist} \PY{o}{=} \PY{n}{ma}\PY{o}{.}\PY{n}{masked\PYZus{}array}\PY{p}{(}\PY{n}{dist}\PY{p}{,} \PY{n}{mask}\PY{o}{=}\PY{n}{np}\PY{o}{.}\PY{n}{identity}\PY{p}{(}\PY{n}{dist}\PY{o}{.}\PY{n}{shape}\PY{p}{[}\PY{l+m+mi}{0}\PY{p}{]}\PY{p}{)}\PY{p}{)}  \PY{c+c1}{\PYZsh{} mask the diagonal}

\PY{n}{disp} \PY{o}{=} \PY{p}{[}\PY{p}{[}\PY{l+s+s2}{\PYZdq{}}\PY{l+s+s2}{movie1}\PY{l+s+s2}{\PYZdq{}}\PY{p}{,} \PY{l+s+s2}{\PYZdq{}}\PY{l+s+s2}{genres}\PY{l+s+s2}{\PYZdq{}}\PY{p}{,} \PY{l+s+s2}{\PYZdq{}}\PY{l+s+s2}{movie2}\PY{l+s+s2}{\PYZdq{}}\PY{p}{,} \PY{l+s+s2}{\PYZdq{}}\PY{l+s+s2}{genres}\PY{l+s+s2}{\PYZdq{}}\PY{p}{]}\PY{p}{]}
\PY{k}{for} \PY{n}{i} \PY{o+ow}{in} \PY{n+nb}{range}\PY{p}{(}\PY{n}{count}\PY{p}{)}\PY{p}{:}
    \PY{n}{min\PYZus{}idx} \PY{o}{=} \PY{n}{np}\PY{o}{.}\PY{n}{argmin}\PY{p}{(}\PY{n}{m\PYZus{}dist}\PY{p}{[}\PY{n}{i}\PY{p}{]}\PY{p}{)}
    \PY{n}{movie1\PYZus{}id} \PY{o}{=} \PY{n+nb}{int}\PY{p}{(}\PY{n}{item\PYZus{}vecs}\PY{p}{[}\PY{n}{i}\PY{p}{,}\PY{l+m+mi}{0}\PY{p}{]}\PY{p}{)}
    \PY{n}{movie2\PYZus{}id} \PY{o}{=} \PY{n+nb}{int}\PY{p}{(}\PY{n}{item\PYZus{}vecs}\PY{p}{[}\PY{n}{min\PYZus{}idx}\PY{p}{,}\PY{l+m+mi}{0}\PY{p}{]}\PY{p}{)}
    \PY{n}{disp}\PY{o}{.}\PY{n}{append}\PY{p}{(} \PY{p}{[}\PY{n}{movie\PYZus{}dict}\PY{p}{[}\PY{n}{movie1\PYZus{}id}\PY{p}{]}\PY{p}{[}\PY{l+s+s1}{\PYZsq{}}\PY{l+s+s1}{title}\PY{l+s+s1}{\PYZsq{}}\PY{p}{]}\PY{p}{,} \PY{n}{movie\PYZus{}dict}\PY{p}{[}\PY{n}{movie1\PYZus{}id}\PY{p}{]}\PY{p}{[}\PY{l+s+s1}{\PYZsq{}}\PY{l+s+s1}{genres}\PY{l+s+s1}{\PYZsq{}}\PY{p}{]}\PY{p}{,}
                  \PY{n}{movie\PYZus{}dict}\PY{p}{[}\PY{n}{movie2\PYZus{}id}\PY{p}{]}\PY{p}{[}\PY{l+s+s1}{\PYZsq{}}\PY{l+s+s1}{title}\PY{l+s+s1}{\PYZsq{}}\PY{p}{]}\PY{p}{,} \PY{n}{movie\PYZus{}dict}\PY{p}{[}\PY{n}{movie1\PYZus{}id}\PY{p}{]}\PY{p}{[}\PY{l+s+s1}{\PYZsq{}}\PY{l+s+s1}{genres}\PY{l+s+s1}{\PYZsq{}}\PY{p}{]}\PY{p}{]}
               \PY{p}{)}
\PY{n}{table} \PY{o}{=} \PY{n}{tabulate}\PY{o}{.}\PY{n}{tabulate}\PY{p}{(}\PY{n}{disp}\PY{p}{,} \PY{n}{tablefmt}\PY{o}{=}\PY{l+s+s1}{\PYZsq{}}\PY{l+s+s1}{html}\PY{l+s+s1}{\PYZsq{}}\PY{p}{,} \PY{n}{headers}\PY{o}{=}\PY{l+s+s2}{\PYZdq{}}\PY{l+s+s2}{firstrow}\PY{l+s+s2}{\PYZdq{}}\PY{p}{)}
\PY{n}{table}
\end{Verbatim}
\end{tcolorbox}

            \begin{tcolorbox}[breakable, size=fbox, boxrule=.5pt, pad at break*=1mm, opacityfill=0]
\prompt{Out}{outcolor}{24}{\boxspacing}
\begin{Verbatim}[commandchars=\\\{\}]
'<table>\textbackslash{}n<thead>\textbackslash{}n<tr><th>movie1
</th><th>genres                                             </th><th>movie2
</th><th>genres
</th></tr>\textbackslash{}n</thead>\textbackslash{}n<tbody>\textbackslash{}n<tr><td>Save the Last Dance (2001)
</td><td>Drama|Romance                                      </td><td>Mona Lisa
Smile (2003)                             </td><td>Drama|Romance
</td></tr>\textbackslash{}n<tr><td>Wedding Planner, The (2001)
</td><td>Comedy|Romance                                     </td><td>Mr. Deeds
(2002)                                   </td><td>Comedy|Romance
</td></tr>\textbackslash{}n<tr><td>Hannibal (2001)
</td><td>Horror|Thriller                                    </td><td>Final
Destination 2 (2003)                         </td><td>Horror|Thriller
</td></tr>\textbackslash{}n<tr><td>Saving Silverman (Evil Woman) (2001)
</td><td>Comedy|Romance                                     </td><td>Down with
Love (2003)                              </td><td>Comedy|Romance
</td></tr>\textbackslash{}n<tr><td>Down to Earth (2001)
</td><td>Comedy|Fantasy|Romance                             </td><td>Bewitched
(2005)                                   </td><td>Comedy|Fantasy|Romance
</td></tr>\textbackslash{}n<tr><td>Mexican, The (2001)
</td><td>Action|Comedy                                      </td><td>Rush Hour 2
(2001)                                 </td><td>Action|Comedy
</td></tr>\textbackslash{}n<tr><td>15 Minutes (2001)                       </td><td>Thriller
</td><td>Panic Room (2002)                                  </td><td>Thriller
</td></tr>\textbackslash{}n<tr><td>Enemy at the Gates (2001)               </td><td>Drama
</td><td>Kung Fu Hustle (Gong fu) (2004)                    </td><td>Drama
</td></tr>\textbackslash{}n<tr><td>Heartbreakers (2001)
</td><td>Comedy|Crime|Romance                               </td><td>Fun with
Dick and Jane (2005)                      </td><td>Comedy|Crime|Romance
</td></tr>\textbackslash{}n<tr><td>Spy Kids (2001)
</td><td>Action|Adventure|Children|Comedy                   </td><td>Tuxedo, The
(2002)                                 </td><td>Action|Adventure|Children|Comedy
</td></tr>\textbackslash{}n<tr><td>Along Came a Spider (2001)
</td><td>Action|Crime|Mystery|Thriller                      </td><td>Insomnia
(2002)                                    </td><td>Action|Crime|Mystery|Thriller
</td></tr>\textbackslash{}n<tr><td>Blow (2001)                             </td><td>Crime|Drama
</td><td>25th Hour (2002)                                   </td><td>Crime|Drama
</td></tr>\textbackslash{}n<tr><td>Bridget Jones\&\#x27;s Diary (2001)
</td><td>Comedy|Drama|Romance                               </td><td>Punch-Drunk
Love (2002)                            </td><td>Comedy|Drama|Romance
</td></tr>\textbackslash{}n<tr><td>Joe Dirt (2001)
</td><td>Adventure|Comedy|Mystery|Romance                   </td><td>Polar
Express, The (2004)
</td><td>Adventure|Comedy|Mystery|Romance
</td></tr>\textbackslash{}n<tr><td>Crocodile Dundee in Los Angeles (2001)
</td><td>Comedy|Drama                                       </td><td>Bewitched
(2005)                                   </td><td>Comedy|Drama
</td></tr>\textbackslash{}n<tr><td>Mummy Returns, The (2001)
</td><td>Action|Adventure|Comedy|Thriller                   </td><td>Rundown,
The (2003)
</td><td>Action|Adventure|Comedy|Thriller
</td></tr>\textbackslash{}n<tr><td>Knight\&\#x27;s Tale, A (2001)
</td><td>Action|Comedy|Romance                              </td><td>Legally
Blonde (2001)                              </td><td>Action|Comedy|Romance
</td></tr>\textbackslash{}n<tr><td>Shrek (2001)
</td><td>Adventure|Animation|Children|Comedy|Fantasy|Romance</td><td>Tangled
(2010)                                     </td><td>Adventure|Animation|Children
|Comedy|Fantasy|Romance</td></tr>\textbackslash{}n<tr><td>Moulin Rouge (2001)
</td><td>Drama|Romance                                      </td><td>Notebook,
The (2004)                               </td><td>Drama|Romance
</td></tr>\textbackslash{}n<tr><td>Pearl Harbor (2001)
</td><td>Action|Drama|Romance                               </td><td>Bridget
Jones: The Edge of Reason (2004)           </td><td>Action|Drama|Romance
</td></tr>\textbackslash{}n<tr><td>Animal, The (2001)                      </td><td>Comedy
</td><td>Dumb and Dumberer: When Harry Met Lloyd (2003)     </td><td>Comedy
</td></tr>\textbackslash{}n<tr><td>Evolution (2001)                        </td><td>Comedy|Sci-
Fi                                      </td><td>Behind Enemy Lines (2001)
</td><td>Comedy|Sci-Fi
</td></tr>\textbackslash{}n<tr><td>Swordfish (2001)
</td><td>Action|Crime|Drama                                 </td><td>We Were
Soldiers (2002)                            </td><td>Action|Crime|Drama
</td></tr>\textbackslash{}n<tr><td>Atlantis: The Lost Empire (2001)
</td><td>Adventure|Animation|Children|Fantasy               </td><td>Cloudy with
a Chance of Meatballs (2009)
</td><td>Adventure|Animation|Children|Fantasy
</td></tr>\textbackslash{}n<tr><td>Lara Croft: Tomb Raider (2001)
</td><td>Action|Adventure                                   </td><td>National
Treasure: Book of Secrets (2007)          </td><td>Action|Adventure
</td></tr>\textbackslash{}n<tr><td>Dr. Dolittle 2 (2001)                   </td><td>Comedy
</td><td>Legally Blonde 2: Red, White \&amp; Blonde (2003)       </td><td>Comedy
</td></tr>\textbackslash{}n<tr><td>Fast and the Furious, The (2001)
</td><td>Action|Crime|Thriller                              </td><td>xXx (2002)
</td><td>Action|Crime|Thriller
</td></tr>\textbackslash{}n<tr><td>A.I. Artificial Intelligence (2001)
</td><td>Adventure|Drama|Sci-Fi                             </td><td>Bubba Ho-
tep (2002)                                </td><td>Adventure|Drama|Sci-Fi
</td></tr>\textbackslash{}n<tr><td>Cats \&amp; Dogs (2001)
</td><td>Children|Comedy                                    </td><td>Robots
(2005)                                      </td><td>Children|Comedy
</td></tr>\textbackslash{}n<tr><td>Scary Movie 2 (2001)                    </td><td>Comedy
</td><td>Orange County (2002)                               </td><td>Comedy
</td></tr>\textbackslash{}n<tr><td>Final Fantasy: The Spirits Within
(2001)</td><td>Adventure|Animation|Fantasy|Sci-Fi
</td><td>Madagascar: Escape 2 Africa (2008)
</td><td>Adventure|Animation|Fantasy|Sci-Fi
</td></tr>\textbackslash{}n<tr><td>Legally Blonde (2001)
</td><td>Comedy|Romance                                     </td><td>Serendipity
(2001)                                 </td><td>Comedy|Romance
</td></tr>\textbackslash{}n<tr><td>Score, The (2001)
</td><td>Action|Drama                                       </td><td>Punisher,
The (2004)                               </td><td>Action|Drama
</td></tr>\textbackslash{}n<tr><td>Jurassic Park III (2001)
</td><td>Action|Adventure|Sci-Fi|Thriller                   </td><td>Men in
Black II (a.k.a. MIIB) (a.k.a. MIB 2) (2002)</td><td>Action|Adventure|Sci-
Fi|Thriller                   </td></tr>\textbackslash{}n<tr><td>America\&\#x27;s Sweethearts
(2001)            </td><td>Comedy|Romance
</td><td>Maid in Manhattan (2002)
</td><td>Comedy|Romance
</td></tr>\textbackslash{}n<tr><td>Ghost World (2001)
</td><td>Comedy|Drama                                       </td><td>Station
Agent, The (2003)                          </td><td>Comedy|Drama
</td></tr>\textbackslash{}n<tr><td>Planet of the Apes (2001)
</td><td>Action|Adventure|Drama|Sci-Fi                      </td><td>Day After
Tomorrow, The (2004)                     </td><td>Action|Adventure|Drama|Sci-Fi
</td></tr>\textbackslash{}n<tr><td>Princess Diaries, The (2001)
</td><td>Children|Comedy|Romance                            </td><td>Lake House,
The (2006)                             </td><td>Children|Comedy|Romance
</td></tr>\textbackslash{}n<tr><td>Rush Hour 2 (2001)
</td><td>Action|Comedy                                      </td><td>Mexican,
The (2001)                                </td><td>Action|Comedy
</td></tr>\textbackslash{}n<tr><td>American Pie 2 (2001)                   </td><td>Comedy
</td><td>Rat Race (2001)                                    </td><td>Comedy
</td></tr>\textbackslash{}n<tr><td>Others, The (2001)
</td><td>Drama|Horror|Mystery|Thriller                      </td><td>The
Machinist (2004)
</td><td>Drama|Horror|Mystery|Thriller
</td></tr>\textbackslash{}n<tr><td>Rat Race (2001)                         </td><td>Comedy
</td><td>American Pie 2 (2001)                              </td><td>Comedy
</td></tr>\textbackslash{}n<tr><td>Jay and Silent Bob Strike Back (2001)
</td><td>Adventure|Comedy                                   </td><td>Mexican,
The (2001)                                </td><td>Adventure|Comedy
</td></tr>\textbackslash{}n<tr><td>Training Day (2001)
</td><td>Crime|Drama|Thriller                               </td><td>Frailty
(2001)                                     </td><td>Crime|Drama|Thriller
</td></tr>\textbackslash{}n<tr><td>Zoolander (2001)                        </td><td>Comedy
</td><td>Old School (2003)                                  </td><td>Comedy
</td></tr>\textbackslash{}n<tr><td>Serendipity (2001)
</td><td>Comedy|Romance                                     </td><td>Legally
Blonde (2001)                              </td><td>Comedy|Romance
</td></tr>\textbackslash{}n<tr><td>Mulholland Drive (2001)
</td><td>Crime|Drama|Mystery|Thriller                       </td><td>Prisoners
(2013)                                   </td><td>Crime|Drama|Mystery|Thriller
</td></tr>\textbackslash{}n<tr><td>From Hell (2001)
</td><td>Crime|Horror|Mystery|Thriller                      </td><td>Identity
(2003)                                    </td><td>Crime|Horror|Mystery|Thriller
</td></tr>\textbackslash{}n<tr><td>Waking Life (2001)
</td><td>Animation|Drama|Fantasy                            </td><td>Warm Bodies
(2013)                                 </td><td>Animation|Drama|Fantasy
</td></tr>\textbackslash{}n<tr><td>K-PAX (2001)
</td><td>Drama|Fantasy|Mystery|Sci-Fi                       </td><td>Gosford
Park (2001)                                </td><td>Drama|Fantasy|Mystery|Sci-Fi
</td></tr>\textbackslash{}n</tbody>\textbackslash{}n</table>'
\end{Verbatim}
\end{tcolorbox}
        
    The results show the model will generally suggest a movie with similar
genre's.

    \#\# 6 - Congratulations! You have completed a content-based recommender
system.

This structure is the basis of many commercial recommender systems. The
user content can be greatly expanded to incorporate more information
about the user if it is available. Items are not limited to movies. This
can be used to recommend any item, books, cars or items that are similar
to an item in your `shopping cart'.

    Please click here if you want to experiment with any of the non-graded
code.

Important Note: Please only do this when you've already passed the
assignment to avoid problems with the autograder.

On the notebook's menu, click ``View'' \textgreater{} ``Cell Toolbar''
\textgreater{} ``Edit Metadata''

Hit the ``Edit Metadata'' button next to the code cell which you want to
lock/unlock

Set the attribute value for ``editable'' to:

``true'' if you want to unlock it

``false'' if you want to lock it

\begin{verbatim}
    </li>
    <li> On the notebook’s menu, click “View” > “Cell Toolbar” > “None” </li>
</ol>
<p> Here's a short demo of how to do the steps above: 
    <br>
    <img src="https://lh3.google.com/u/0/d/14Xy_Mb17CZVgzVAgq7NCjMVBvSae3xO1" align="center" alt="unlock_cells.gif">
\end{verbatim}


    % Add a bibliography block to the postdoc
    
    
    
\end{document}
